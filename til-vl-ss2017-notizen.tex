\documentclass[a4paper,10pt]{article}

\usepackage[utf8]{inputenc}
\usepackage[ngerman]{babel}
\usepackage[top=2.5cm,bottom=2.5cm,left=2.5cm,right=2.5cm]{geometry}
\usepackage[T1]{fontenc}
\usepackage{graphicx}
\usepackage[table]{xcolor}
\usepackage{fancyhdr}
\usepackage{tgpagella}
\usepackage{marginnote}
\usepackage{nameref}
\usepackage{amsmath,amssymb,amsfonts}
\usepackage{enumitem}
\usepackage{sectsty}
\usepackage{wrapfig}
\usepackage{listings}
\usepackage[
    pdfauthor={Dominik Pataky},
    pdftitle={Theoretische Informatik und Logik},
    pdfsubject={Notizen und Begriffsklärungen aus dem Sommersemester 2017, TU Dresden},
    colorlinks=true,linkcolor=black,urlcolor=link]{hyperref}

\definecolor{light-gray}{gray}{0.9}
\definecolor{light-red}{RGB}{255,105,105}
\definecolor{vl}{RGB}{84,200,70}
\definecolor{link}{RGB}{84,100,220}
\definecolor{sectionblue}{RGB}{0,64,114}
\definecolor{subsectionblue}{RGB}{0,94,167}
\definecolor{subsubsectionblue}{RGB}{0,110,195}

% set enumeration style
\setlist[enumerate, 1]{label=\textbf{\alph*)}, leftmargin=2em}
\setlist{itemsep=0em}

% Small line under top header
\renewcommand{\headrulewidth}{0.1pt}

% Formatting macros
\newcommand{\authorHead}[1]{\rhead{\footnotesize Dozent: Prof. Krötzsch, Autor: #1, CC BY-SA 4.0}}
\newcommand{\vl}[1]{\colorbox{vl}{\textcolor{white}{\small\textbf{#1}}}}
\newcommand{\f}[1]{\textbf{#1}}
\newcommand{\blank}{\text{\textvisiblespace}}
\newcommand{\hili}[1]{\colorbox{light-gray}{\textcolor{black}{#1}}}
\newcommand{\verweis}[1]{\textit{\ref{#1} \nameref{#1}}}
\newcommand{\LOES}{\f{Lösung:~}}
% Problems
\newcommand{\prob}[1]{\textbf{#1}}
\newcommand{\prspec}[1]{\prob{P}_{\text{#1}}}
\newcommand{\phalt}{\prspec{halt}}
\newcommand{\paq}{\prspec{äquiv}}
% Math sets
\newcommand{\N}{\mathbb{N}}
\newcommand{\Q}{\mathbb{Q}}
\newcommand{\R}{\mathbb{R}}
\newcommand{\POT}{\mathcal{P}}
% Shortcuts
\newcommand{\TMM}[1]{\mathcal{M}_{#1}}
\newcommand{\LANG}{\mathcal{L}}
\newcommand{\SIGS}{\Sigma^{*}}
% Math texts
\newcommand{\MT}[1]{\mathit{#1}}
\newcommand{\NP}{\MT{NP}}
% Plogik
\newcommand{\INT}{\mathcal{I}}
\newcommand{\ZUW}{\mathcal{Z}}
\newcommand{\FM}{\mathcal{T}}
\newcommand{\HU}{\Delta_{F}}
\newcommand{\HE}{\MT{HE}}
\newcommand{\GDW}{\Leftrightarrow}
% Uebungen
\newcommand{\C}{\mathcal{C}}
\newcommand{\SESP}{\MT{SETSPLITTING}}
\newcommand{\SUBINT}{\mathcal{J}}
% Musterklausur
\newcommand{\copaq}{\overline{\mbox{\prob{P}\strut}}_{\text{äquiv}}}
\newcommand{\M}{\mathcal{M}}
\newcommand{\POINT}{{\textcolor{red}{* }}}


\begin{document}
    % Header and footer styles
    \pagestyle{fancy}
    \lhead{\footnotesize Theoretische Informatik und Logik SS2017}
    \chead{}
    \authorHead{Pataky}
    \lfoot{}
    \cfoot{\thepage}
    \rfoot{}

    \begin{titlepage}
        \centering

        {\scshape\Large Notizen zur Vorlesung \par}
        {\scshape\LARGE Theoretische Informatik und Logik \par}

        \tableofcontents
        \vfill

        \begin{abstract}
            Notizen zur Vorlesung \url{https://iccl.inf.tu-dresden.de/web/Theoretische_Informatik_und_Logik_(SS2017)}. \LaTeX-Quellen unter \url{https://github.com/cooox/theolog-ss2017}.
            Die Markierungen verweisen jeweils auf die Vorlesungsnummer in \vl{FS} bzw. \vl{TIL}.
            Obwohl der Schwerpunkt auf TheoLog liegt, habe ich ein paar Definitionen aus Formale Systeme mit einbezogen, da TheoLog diese weiterverwendet. \\
            Einige Formulierungen habe ich aus den hervorragenden Folien von Prof. Krötzsch geliehen. Quellen dieser Folien sind auf Github zu finden unter \url{https://github.com/mkroetzsch/TheoLog} und sind unter der Lizenz CC BY 3.0 DE verwendbar. Für diese gilt: „(C) Markus Krötzsch, \url{https://iccl.inf.tu-dresden.de/web/TheoLog2017}, CC BY 3.0 DE“. \\
            Die Lösungen für die Übungen der Prädikatenlogik sowie Rep 3 und der Musterklausur wurden beigetragen von Tim Schmittmann, \url{https://github.com/TimSchmittmann}. \\
            Lizenz für Übungsaufgaben: „\textcopyright\ 2017 Monika Sturm, Daniel Borchmann. This work is licensed under the Creative Commons Attribution-ShareAlike 4.0 International License“. Siehe Quellen in \url{https://github.com/mkroetzsch/TheoLog/tree/master/Uebungen}.
        \end{abstract}

        \vfill
        \begin{tabular}{p{3cm} p{10cm}}
            Autor & Dominik Pataky \\
            Dozent & Prof. Markus Krötzsch \\
            Ort & Fakultät Informatik, TU Dresden \\
            Zeit & Sommersemester 2017 \\
            Letztes Update & \today \\
            Lizenz & CC BY-SA 4.0
        \end{tabular}

    \end{titlepage}

    \setlength\parindent{0cm} % indentation of paragraphs
    \sectionfont{\color{sectionblue}\Large}
    \subsectionfont{\color{subsectionblue}\large}
    \subsubsectionfont{\color{subsubsectionblue}\normalsize}

    \include{includes/formale-systeme}
    \section{Theoretische Informatik}

Die Theoretische Informatik beginnt mit der Berechenbarkeitstheorie. Hier nutzen wir das Maschinenmodell der Turingmaschine, um Aussagen über die Entscheidbarkeit von Problemen zu treffen. Die Berechenbarkeitstheorie klassifiziert Probleme demnach nach ihrer Berechenbarkeit bzw. Entscheidbarkeit. \\

Ab Kapitel \ref{subsec:komplex} behandeln wir die Komplexitätstheorie. In dieser geht es um die Fragestellung, wie viel Zeit und Speicher ein entscheidbares und algorithmisch formalisierbares Problem benötigt, um von einer Maschine wie der Turingmaschine gelöst zu werden. Die Komplexitätstheorie klassifiziert also Probleme, von denen wir wissen, dass sie berechenbar/entscheidbar sind, nach ihrem Ressourcenaufwand.

\subsection{Allgemeines}
    \begin{description}
        \item[Surjektiv, Injektiv, Bijektiv] Betrifft Abbildungen zwischen zwei Mengen. \f{Surjektiv} bedeutet, jedes Element in der Zielmenge wird mindestens einmal getroffen. \f{Injektiv} bedeutet, jedes Element in der Zielmenge wird höchstens einmal getroffen. \f{Bijektiv} bedeutet, die Abbildungen sind sowohl surjektiv als auch injektiv, es besteht eine eindeutige Zuordnung. \textit{Merkhilfe: Paare „(Injektiv höchstens), (mindestens surjektiv)“ $\to$ IHMS in alphabetischer Reihenfolge}

        \item[Kardinalität, Mächtigkeit] Die Menge der Elemente in einer Menge. \\
            Eine unendliche Zielmenge ist \f{abzählbar}, wenn es eine bijektive Abbildung von der Menge $\N$ auf die Zielmenge gibt. Ist die Zielmenge endlich, ist sie natürlich ebenfalls abzählbar. \f{Überabzählbare} Mengen hingegen haben mehr Elemente, als Elemente in $\N$ sind.
    \end{description}

\subsection{Turingmaschinen}
    \begin{description}
        \item[Turingmaschine] deterministisch als DTM oder nichtdeterministisch als NTM. \\
            Definiert als Tupel $(Q,\Sigma,\Gamma,\delta,q_0,F)$ mit endlicher Menge von Zuständen $Q$, Eingabealphabet $\Sigma$, Arbeitsalphabet $\Gamma$, Übergangsfunktion $\delta$, Startzustand $q_0$ und Menge von akzeptierenden Endzuständen $F$. Können ein oder mehrere Bänder haben. Siehe auch Church-Turing-These. \vl{FS 18} \vl{TIL 1}

            \begin{description}
                \item[Funktion] Turingmaschine kann eine Funktion von Eingaben auf Ausgabewörter definieren. Wenn eine TM bei Eingabe $w$ anhält und die Ausgabe der Form $v\blank\blank\dots$ entspricht, hat diese TM die Funktion berechnet.
                \item[Sprache] die von einer Turingmaschine erkannte Sprache ist die Menge aller Wörter, die von dieser TM akzeptiert werden (d.h. in einem Endzustand hält).

                \item[Konfiguration]
                    der „Gesamtzustand“ einer TM, bestehend aus Zustand, Bandinhalt und Position des Lese-/Schreibkopfs;
                    geschrieben als Wort (Bandinhalt), in dem der Zustand vor der Position des Kopfes eingefügt ist. Beispiel $ \blank\blank q_0aaba \blank\blank$.
                \item[Übergangsrelation]
                    Beziehung zwischen zwei Konfigurationen wenn die TM von der ersten in die zweite übergehen kann
                    (deterministisch oder nichtdeterministisch)
                \item[Lauf] mögliche Abfolge von Konfigurationen einer TM, beginnend mit der Startkonfiguration; kann endlich oder unendlich sein
                \item[Halten] Ende der Abarbeitung, wenn die TM in einer Konfiguration keinen Übergang mehr zur Verfügung hat.

                \item[Transducer] Ausgabe der Turingmaschine ist Inhalt des Bandes, wenn TM hält, ansonsten undefiniert. Endzustände sind irrelevant.
                \item[Entscheider] Ausgabe der Turingmaschine ist „Akzeptiert“, wenn TM in Endzustand hält, ansonsten „verwirft“ (beinhaltet auch „TM hält nicht“). Bandinhalt ist irrelevant.
                \item[Aufzähler] ist eine DTM, die bei Eingabe des leeren Bandes immer wieder (d.h. bis zum letzten Wort bei endlichen Sprachen) einen Zustand $q_{Ausgabe}$ erreicht, in welchem das aktuelle Band ein Wort aus der Sprache dieser DTM ist. Die Sprache dieser DTM ist dann die Menge der so erzeugten Wörter. Diese DTM muss nicht halten, die Sprache kann unendlich sein. Wörter dürfen mehrfach ausgegeben werden.

                \item[Universalmaschine $U$] eine Turingmaschine, die andere TM als Eingabe kodiert erhält und diese simuliert. Die Kodierung ist dabei z.B. binär, mit dem Trennsymbol $\#$. Hat vier Bänder: Eingabeband von $U$ mit kodierter TM und kodierter Eingabe $w$, Arbeitsband von $U$, Band 3 mit aktuellem Zustand der simulierten TM und Band 4 als Arbeitsband der simulierten TM. \\ Für die Arbeitsweise siehe \vl{TIL 4}
            \end{description}

        \item[Berechenbarkeit] bezogen auf Funktionen. Eine Funktion $F$ heißt berechenbar, wenn es eine DTM gibt, die $F$ berechnet. Ist durch geeignete Kodierung (z.B. binär) erweiterbar auf natürliche Zahlen, Wörterlisten und andere Mengen. \vl{TIL 2}
            \begin{description}
                \item[rekursiv] eine berechenbare totale Funktion ist rekursiv.
                \item[partiell rekursiv] eine berechenbare partielle Funktion ist partiell rekursiv.
            \end{description}

        \item[Entscheidbarkeit] bezogen auf Sprachen. \vl{TIL 2}
            \begin{description}
                \item[entscheidbar / berechenbar / rekursiv] es existiert eine Turingmaschine, die das Wortproblem der Sprache entscheidet. D.h. die Turingmaschine ist Entscheider und die Sprache ist gleich der Sprache der TM.
                \item[semi-entscheidbar / Turing-erkennbar / Turing-akzeptierbar / rekursiv aufzählbar] es existiert eine Turingmaschine, deren erzeugte Sprache gleich der Sprache ist, jedoch die TM kein Entscheider ist. \\
                Eine Sprache ist genau dann semi-entscheidbar, wenn es einen Aufzähler für diese Sprache gibt.
                \item[unentscheidbar] sonst. \\
                    „Es gibt Sprachen und Funktionen, die nicht berechenbar sind.“ Beweis anhand der abzählbaren Menge von Turingmaschinen im Vergleich zur Überabzählbarkeit der Menge der Sprachen über jedem Alphabet.
            \end{description}

        \item[Probleme] der Kategorie „Unentscheidbar bzw. unberechenbar, nicht berechenbar“. \vl{TIL 2}
            \begin{description}
                \item[Busy-Beaver-Funktion] ist nicht berechenbar und wächst sehr schnell. Die Funktion nimmt eine natürliche Zahl $n$ und gibt die maximale Anzahl $x$-Symbole, welche eine DTM mit $n$ Zuständen und dem Arbeitsalphabet $\{x,\blank\}$ bis zu ihrem Halt schreiben kann, zurück.
            \end{description}
    \end{description}

\subsection{LOOP und WHILE}
    LOOP und WHILE sind eine Erfindung von Schöning und sind quasi eine pädagogische Brücke zwischen den Ultra-low-level Turingmaschinen und High-level Programmiersprachen. WHILE baut auf LOOP auf. \vl{TIL 3}
    \begin{description}
        \item[LOOP] Besteht aus Variablen, Wertzuweisungen und Schleifen. Die Eingabe einer Menge von natürlichen Zahlen wird in $x_1, x_2, …$ gespeichert. Die Ausgabe ist eine natürliche Zahl, gespeichert in $x_0$. Alle weiteren Variablen haben den Wert $0$. LOOP terminiert immer in endlich vielen Schritten. Berechnet eine totale Funktion.
            \begin{description}
                \item[Variablen] Menge $\{x_0,x_1,…\}$ oder auch $\{x, y, \mathit{myVariable}\}$. Haben als Wert eine natürliche Zahl.
                \item[Wertzuweisungen] in der Form $x := y + n$ oder $x := y - n$, wobei $n$ eine natürliche Zahl ist. Eine Wertzuweisung ist bereits ein LOOP-Programm.
                \item[Schleifen] in der Form LOOP $x$ DO $P$ END, wobei $P$ wieder ein LOOP-Programm ist. Der Wert der Variable $x$ kann in $P$ nicht geändert werden. Daher terminiert ein LOOP-Programm immer in endlich vielen Schritten.
                \item[Hintereinanderausführung] wenn $P_0$ und $P_1$ LOOP-Programme, dann auch $P_0;P_1$.
                \item[Syntax-Erweiterung] Die Syntax lässt sich zur Vereinfachung erweitern.
                    \begin{description}
                        \item[Wertzuweisung \hili{$x:=y$}] $x:=y+0$
                        \item[Rücksetzen \hili{$x:=0$}] LOOP $x$ DO $x:=x-1$ END
                        \item[Wertzuweisung Zahl \hili{$x:=n$}] $x:=0;x:=x+n$. Alternativ $x:=null+n$
                        \item[Variablen-Addition \hili{$x:=y+z$}] $x:=y;$ LOOP $z$ DO $x:=x+1$ END
                        \item[Bedingung \hili{IF $x\neq0$ THEN}] LOOP $x$ DO $y:=1$ END $;$ LOOP $y$ DO $P$ END
                    \end{description}
                \item[Berechenbarkeit] eine Funktion heißt LOOP-berechenbar, wenn es ein LOOP-Programm gibt, welches die Funktion berechnet. Auch hier ist mit geeigneter Kodierung wieder mehr machbar, als nur die natürlichen Zahlen in Betracht zu ziehen (Beispiel Wortproblem, Probleme in NP, gängige Algorithmen). Es gibt berechenbare totale Funktionen, die nicht LOOP-berechenbar sind (vgl. Ackermannfunktion).
            \end{description}

\newpage
        \item[WHILE] Basiert auf LOOP und erweitert dieses. Jedes LOOP-Programm ist auch ein WHILE-Programm.
            \begin{description}
                \item[Schleifen] in der Form WHILE $x \neq 0$ DO $P$ WHEN, wobei $P$ wieder WHILE-Programm. Im Gegensatz zu LOOP kann in WHILE der Wert von $x$ in $P$ zur Laufzeit geändert werden. Es kann also passieren, dass das Programm nicht terminiert wenn $x$ nie auf $0$ gesetzt wird.
                \item[Konvertierung] LOOP-Schleifen können in WHILE-Schleifen konvertiert werden. Eine DTM kann WHILE-Programme simulieren und ein WHILE-Programm DTMen simulieren.
                \item[Berechenbarkeit] Eine partielle Funktion heißt WHILE-berechenbar, wenn es ein WHILE-Programm gibt, welches bei einem definierten $f(n_0,n_1,…)$ terminiert und bei einem nicht definierten Wertebereich nicht terminiert. Wenn eine partielle Funktion WHILE-berechenbar ist, ist sie \f{Turing-berechenbar}.
            \end{description}
    \end{description}


\subsection{Unentscheidbare Probleme und Reduktionen}
    Beweis durch Diagonalisierung, Reduktionen \vl{TIL 4}
    \begin{description}
        \item[Probleme] der Kategorie „unentscheidbar“.
        \begin{description}
            \item[Halteproblem $\phalt$] Semi-entscheidbar. Frage: „Gegeben eine Turingmaschine $M$ und ein Wort $w$. Wird die Turingmaschine $M$ für die Eingabe $w$ jemals anhalten?“. Das Halteproblem $\phalt$ der Turingmaschine $M$ für das Wort $w$ kann formal kodiert werden als $enc(M)\#\#enc(w)$ und einer universellen Turingmaschine zur Überprüfung übergeben werden. Beweise für Unentscheidbarkeit anhand Diagonalisierung und Reduktion in \vl{TIL 4}

            \item[Goldbachsche Vermutung] Beispiel für ein auf das Halteproblem reduzierbares Problem. Besagt, dass jede gerade Zahl $n \ge 4$ die Summe zweier Primzahlen ist. Zum Beispiel ist $4 = 2 + 2$ und $100 = 47 + 53$. Lässt man nun eine Turingmaschine diese Vermutung systematisch beginnend bei $4$ testen, würde ein Anhalten bei Misserfolg $\phalt$ und „die Vermutung stimmt nicht“ gleichzeitig lösen. Gäbe es demnach ein Programm, welches $\phalt$ lösen kann (entscheidet), wäre eine separate Überprüfung der Goldbachschen Vermutung nicht nötig. Die Frage der Goldbachschen Vermutung wäre sofort beantwortet.

            \item[$\epsilon$~-Halteproblem] „Gegeben sei eine Turingmaschine. Wird diese TM für die leere Eingabe $\epsilon$ jemals anhalten?“. Unentscheidbar.
        \end{description}

        \item[Beweismethoden] zum Nachweis der Unentscheidbarkeit.
        \begin{description}
            \item[Kardinalität] Beweis von Aussagen anhand der unterschiedlichen Kardinalitäten.
            \item[Diagonalisierung] Berechenbarkeit annehmen und einen paradoxen Algorithmus für das Problem konstruieren.
            \item[Reduktion] Reduktion (Rückführung) eines Problems auf ein anderes Problem (Einbetten eines Problems in ein anderes). Die Reduktion ist ein Entscheid\-bar\-keits\-algorithmus. Entscheidbarkeit, Semi-Entscheidbarkeit, Co-Semi-Entscheidbarkeit werden übertragen („wenn A semi-entscheidbar und es gilt $B \leq_{m} A$, dann auch B semi-entscheidbar“). Siehe auch \verweis{subsec:fs-komplexitaet}. \vl{TIL 4}
                \begin{description}
                    \item[Turing-Reduktion] Ein Problem \prob{P} ist Turing-reduzierbar auf ein Problem \prob{Q} (in Symbolen: $\prob{P} \leq_T\prob{Q}$), wenn man \prob{P} mit einem Programm lösen könnte, welches ein Programm für \prob{Q} als Unterprogramm (auch: Subroutine) aufrufen darf. Das Programm für \prob{Q} muss hierbei nicht existieren.

                    \item[Many-One-Reduktion] Eine berechenbare totale Funktion $f: \Sigma^* \to \Sigma^*$ ist eine Many-One-Reduktion von einer Sprache \prob{P} auf eine Sprache \prob{Q} (in Symbolen: $\prob{P} \leq_m \prob{Q}$), wenn für alle Wörter $w \in \Sigma^*$ gilt: $w \in \prob{P}$ gdw. $f(w) \in \prob{Q}$. \\
                    Schwächer als Turing-Reduktion, jede Many-One-Reduktion kann als Turing-Reduktion ausgedrückt werden (dies gilt jedoch nicht andersherum).

                    \item[Polynomielle Many-One-Reduktion] Many-One-Reduktion mit einer polynomiell berechenbaren Funktion. In Symbolen: $\prob{P} \leq_{p} \prob{Q}$. Bedeutet: „Q ist mindestens genauso schwer wie P“. \vl{TIL 8}
                \end{description}
            \item[Satz von Rice] Siehe \verweis{subsec:fs-sprachen-automaten}. \vl{TIL 5} \\
                „Praktisch alle interessanten Fragen zu Sprachen von Turingmaschinen sind unentscheibar“. \\
                Eingabe ist eine Turingmaschine, Ausgabe „hat die Sprache der TM die Eigenschaft?“.
        \end{description}
    \end{description}


\newpage
\subsection{Semi-Entscheidbarkeit}
    \textit{Hinweis: Hierzu gibt es im Schöning gute graphische Darstellungen}. \vl{TIL 5}
    \begin{description}
        \item[Komplement] einer Sprache $L$: $\overline{L} = \{w \in \Sigma^* ~|~ w \notin L \}$ (Achtung: auf Kontext achten. Komplement des Halteproblems ist z.B. anderer Form). Die Turing-Reduktionen $\overline{L} \leq_T L$ bzw. $L \leq_T \overline{L}$ sind mit einer Turingmaschine überprüfbar. Für eine Eingabe $w$ entscheidet diese, ob $w \in L$ und invertiert das Ergebnis.
        \item[Semi-Entscheidbarkeit] Beispiel anhand des Halteproblems: simuliere eine Turingmaschine und deren Eingabe, kodiert als $enc(M)\#\#enc(w)$. Wenn $M$ hält, hält auch die universelle Turingmaschine und akzeptiert. Eine Sprache $L$ ist entscheidbar, wenn sowohl $L$ als auch $\overline{L}$ semi-entscheidbar sind.
        \item[Co-Semi-Entscheidbarkeit] Wenn $L$ nicht semi-entscheidbar, aber $\overline{L}$ semi-entscheidbar, dann ist $L$ co-semi-entscheidbar. Wenn eine Sprache $L$ unentscheidbar, jedoch semi-entscheidbar ist, kann $\overline{L}$ nicht semi-entscheidbar sein.
    \end{description}


\subsection{Postsches Korrespondenzproblem}
    Auch: \f{PCP}. Ein unentscheidbares Problem ohne direkten Bezug zu einer Berechnung. \vl{TIL 5}
    \begin{description}
        \item[PCP] Bei diesem Problem nimmt man eine Reihe von 2-Tupeln (anschaulich vergleichbar mit Dominosteinen) mit je einem Wert oben und einem unten. Ziel der Lösung ist nun, die gegebenen Tupel so anzuordnen, dass oben und unten die gleiche Wortkette entsteht. Beispiel: wir haben die drei Tupel (AB, B), (B, BBB) und (BB, BA). Eine Anordnung mit zehn Tupeln ergibt dann die Lösung. Es kann vorkommen, dass das Problem keine Lösung besitzt.
        \item[MPCP] Hilfskonstruktion. Wir nutzen MPCP, um das Halteproblem auf MPCP zu reduzieren. Folgend reduzieren wir MPCP auf PCP. Beim MPCP wird PCP verwendet, jedoch das Start-Tupel vorgegeben. Die Lösung eines MPCP ist auch eine Lösung des entsprechenden PCP, welche mit dem gegebenen Start-Tupel beginnt.
    \end{description}


\subsection{Unentscheidbare Probleme formaler Sprachen}
    In diesem Kapitel wird wieder auf \verweis{subsec:fs-sprachen-automaten} zurückgegriffen. Eine durch eine Grammatik $G$ erzeugte Sprache wird als $L(G)$ bezeichnet. Für Beweise der folgenden Sätze siehe Vorlesung. Siehe auch Chomsky-Hierarchie in \verweis{subsec:fs-sprachen-automaten}. \vl{TIL 6}
    \begin{itemize}
        \setlength\itemsep{0em}
        \item Das Schnittproblem regulärer Grammatiken (Typ 3) ist entscheidbar.
        \item Das Schnittproblem kontextfreier Grammatiken (Typ 2, \f{CFG}) ist unentscheidbar. Beweis durch Many-One-Reduktion vom PCP.
        \item Das Leerheitsproblem für kontextfreie Grammatiken ist entscheidbar.
        \item Kontextfreie Sprachen sind unter Vereinigung abgeschlossen.
        \item Deterministische kontextfreie Sprachen sind unter Komplement abgeschlossen.
        \item Das Äquivalenzproblem kontextfreier Grammatiken ist unentscheidbar.
    \end{itemize}

\subsection{Komplexitätstheorie}
\label{subsec:komplex}
    Untersuchung von Problemkomplexitäten und Suche nach Methoden zur Bestimmung der Komplexität eines Problems. Klassierung zwischen „leicht lösbar“ bis „schwer lösbar“. Einteilung von berechenbaren Problemen entsprechend der Menge an Ressourcen, die zu ihrer Lösung nötig sind. Einführung anhand von Beispielen. \vl{TIL 7}
    \begin{description}
        \item[Eulerpfad] Ein Eulerpfad ist ein Pfad in einem Graphen, der jede Kante genau einmal durchquert. Ein Eulerkreis ist ein zyklischer Eulerpfad. Ein Graph hat genau dann einen Eulerschen Pfad, wenn er maximal zwei Knoten ungeraden Grades besitzt und zusammenhängend ist. \vl{TIL 7}

        \item[Schranken von Turingmaschinen] in Zeit (Berechnungsschritte) und Raum (Speicherzellen). \\ Siehe \verweis{subsec:fs-komplexitaet}.
        \item[$O$-Notation] Siehe \verweis{subsec:fs-komplexitaet}.

\newpage
        \item[Komplexitätsklassen] erfassen Sprachklassen je nach ihrer Komplexität. \vl{FS 24}
            \begin{description}
                \item[PTime (P)] DTM mit polynomieller Zeit. \\
                    Unter Komplement abgeschlossen, dazu für $L$ akzeptierende Zustände für $\overline{L}$ invertieren.
                \item[ExpTime (Exp)] DTM mit exponentieller Zeit.
                \item[LogSpace (L)] DTM mit logarithmischem Speicher.
                \item[PSpace] DTM mit polynomiellem Speicher. \\

                \item[NPTime (NP)] NTM mit polynomieller Zeit.
                \item[coNP] Klasse der Sprachen $L$, für die $\overline{L} \in \NP$ gilt. Vermutung: $\MT{coNP} \neq \NP$
                \item[NExpTime (NExp)] NTP mit exponentieller Zeit.
                \item[NLogSpace (NL)] NTM mit logarithmischem Speicher. $\MT{NL} = \MT{coNL}$ nach Satz von Immerman, Sz.
                \item[NPSpace] NTM mit polynomiellem Speicher (ist gleich PSpace, nach Satz von Savitch).
            \end{description}

        \item[NP-vollständige Probleme] Auf diese kann reduziert werden, um zu zeigen, dass Sprache $\in \NP$. \vl{TIL 9}
            \begin{description}
                \item[SAT] siehe \verweis{subsec:fs-komplexitaet}.
                \item[Hamiltonpfad] Ein Hamiltonpfad ist ein Pfad in einem Graphen, der jeden Knoten genau einmal durchquert. Ein Hamiltonkreis ist ein zyklischer Hamiltonpfad. \vl{TIL 11}
                \item[Clique] Eine Clique ist ein Graph, bei dem jeder Knoten mit jedem anderen direkt durch eine Kante verbunden ist.
                    Gegeben: Ein Graph $G$ und eine Zahl $k$. Frage: Enthält $G$ eine Clique mit $k$ Knoten?
                \item[Unabhängige Mengen] Eine unabhängige Menge ist eine Teilmenge von Knoten in einem Graph, bei der kein Knoten mit einem anderen direkt verbunden ist.
                    Gegeben: Ein Graph $G$ und eine Zahl $k$. Frage: Enthält $G$ eine unabhängige Menge mit $k$ Knoten?
                \item[Teilmengen-Summe (subset sum)]  Gegeben: Eine Menge von Gegenständen $S = \{\ a_{1}, \dots, a_{n} \}$ , wobei jedem Gegenstand $a_{i}$ ein Wert $v(a_{i})$ zugeordnet ist; eine gewünschte Zahl $z$. Frage: Gibt es eine Teilmenge $T \subseteq S$ mit $\sum\limits_{a \in T} v(a) = z$?
                \item[Rucksack (knapsack)] Gegeben: Eine Menge von Gegenständen $G = \{\ a_{1}, \dots, a_{n}\ \}$, wobei jedem Gegenstand $a_{i}$ ein Wert $v(a_{i})$ und ein Gewicht $g(a_{i})$ zugeordnet ist; ein Mindestwert $w$ und ein Gewichtslimit $l$. \\
                    Frage: Gibt es eine Teilmenge $T \subseteq G$, so dass (1) $\sum\limits_{a \in T} g(a) \leq l$ und (2) $\sum\limits_{a \in T} v(a) \geq w$, d.h. eine Auswahl Gegenstände, deren Wert größer $w$ ist, jedoch das gesamte Gewicht $l$ nicht überschreitet?
            \end{description}

        \item[Pseudopolynomielle Probleme] Die Probleme Teilmengen-Summe und Rucksack kann man bei \f{unärer Kodierung} in pseudopolynomieller Zeit durch z.B. dynamische Programmierung lösen. Probleme, welche selbst dann noch NP-vollständig sind, wenn man alle Zahlen unär kodiert, heißen stark NP-vollständig. \vl{TIL 10}

        \item[NL-vollständige Probleme] Können mit einer NTM und logarithmischem Speicher gelöst werden.
            \begin{description}
                \item[Erreichbarkeit] in gerichteten Graphen. Gegeben: gerichteter Graph $G$ mit Knoten $s$ und $t$. Frage: Gibt es in $G$ einen gerichteten Pfad von $s$ nach $t$?
            \end{description}

        \item[QBF] Eine Quantifizierte Boolsche Formel (QBF) ist eine logische Formel der folgenden Form: \\
            $Q_{1}p_{1}.Q_{2}p_{2}.\dots.Q_{l}p_{l}.F[p_{1},\dots,p_{l}]$ mit $i \geq 0, Q_{i} \in \{\ \exists, \forall\ \}$ Quantoren, $p_{i}$ aussagenlogischen Atomen (Variablen) und $F$ einer aussagenlogischen Formel mit Atomen $p_{1},\dots,p_{l}$. \vl{TIL 11}

        \item[PSpace-Probleme] PSpace-hart und -vollständig. Siehe auch Satz von Savitch. \vl{TIL 11} \vl{TIL 12}
            \begin{description}
                \item[TrueQBF] PSpace-vollständig. Gegeben: eine QBF $Q$. Frage: Ist $W(Q) = 1$? \\
                    Es gilt $\MT{SAT} \leq_{p} \MT{TrueQBF}$; eine Tautologie lässt sich auf TrueQBF reduzieren.
                \item[TrueQBF$_{\text{alt}}$] Ist TrueQBF mit alternierenden All- bzw. Existenzquantoren.
                \item[Geography] PSpace-vollständig. Gegeben: Ein gerichteter Graph und ein Startknoten. Frage: Gibt es eine Gewinnstrategie für dieses TrueQBF-Spiel Geography?
            \end{description}

        \item[Linear Speedup Theorem] Sei $M$ eine Turingmaschine mit $k > 1$ Bändern, die bei Eingaben der Länge $n$ nach maximal $f(n)$ Schritten hält. Dann gibt es für jede natürliche Zahl $c > 0$ eine äquivalente $k$-Band Turingmaschine $M'$, die nach maximal $\frac{f(n)}{c} + n + 2$ Schritten hält. \\
        Bedeutet: in der Theorie kann jedes Programm mit Hilfe mehrerer Bänder „beliebig schneller“ gemacht werden. Dies ist praktisch nicht umsetzbar, da eine Turingmaschine nicht beliebig große Datenmengen in einem Schritt lesen und nicht beliebig komplexe Zustandsübergänge in konstanter Zeit realisieren kann.

        \item[Polynomieller Verifikator, Zertifikate] Ein polynomieller Verifikator für eine Sprache $L \subseteq \SIGS$ ist eine polynomiell-zeitbeschränkte, deterministische TM $M$, für die gilt:
            \begin{itemize}
                \item $M$ akzeptiert nur Wörter der Form $w\#z$ mit: $w \in L$, $z \in \SIGS$ ist ein \f{Zertifikat} polynomieller Länge (d.h. für $M$ gibt es ein Polynom $p$ mit $|z| \leq p(|w|)$)

                \item Für jedes Wort $w \in L$ gibt es ein solches Wort $w\#z \in L(M)$.
            \end{itemize}
            Das Zertifikat $z$ kodiert die Lösung des Problems $w$, die der Verifikator lediglich nachprüft. Zertifikate sollten kurz sein, damit die Prüfung selbst nicht länger dauert als die Lösung des Problems. Zertifikate werden auch \f{Nachweis}, \f{Beweis} oder \f{Zeuge} genannt. \vl{TIL 9}

        \item[Satz von Ladner] Falls $P \neq \NP$, dann gibt es Probleme in NP, die weder NP-vollständig sind noch in P liegen. Diese Probleme heißen NP-intermediate. \vl{TIL 10}

        \item[Komplexität und Spiele] Spiele lassen sich in verschiedene Klassen einsortieren. \vl{TIL 12}
            \begin{description}
                \item[NP] Solitaire, Sudoku, Minesweeper, Tetris
                \item[PSpace] Geography, Reversi, Tic-Tac-Toe (Spiele, bei denen zwei Spieler abwechseln ziehen)
                \item[ExpTime] Schach, Dame, Go, Stern-Halma (Züge können rückgängig gemacht werden)
            \end{description}
    \end{description}


\subsection{Eigenschaften der Komplexitätsklassen}
    Siehe \verweis{subsec:fs-komplexitaet} für eine Übersicht der Klassen. \vl{TIL 7}
    \begin{description}
        \item[Nichtdeterministische Klassen] $NL \subseteq NP \subseteq NPSpace \subseteq NExp$
        \item[DTM auch als NTM, d.h. nichtdet. stärker] $L \subseteq NL$, $P \subseteq NP$, $\mathit{PSpace} \subseteq \mathit{NPSpace}$, $Exp \subseteq NExp$
        \item[Satz von Savitch] Speicherbeschränkte NTM können durch DTMs nur mit  quadratischen Mehrkosten simuliert werden. Insbesondere gilt damit $\mathit{PSpace} = \mathit{NPSpace}$.
    \end{description}

    Zusammenfassend: $L \subseteq NL \subseteq P \subseteq NP \subseteq \mathit{PSpace} = \mathit{NPSpace} \subseteq Exp \subseteq \MT{NExp}$. \\
    Jedoch ist zu beachten:

    \begin{itemize}
        \setlength\itemsep{0em}
        \item Wir wissen nicht, ob irgendeines dieser $\subseteq$ sogar $\subsetneq$ ist.
        \item Insbesondere wissen wir nicht, ob $P \subsetneq NP$ oder $P = NP$.
        \item Wir wissen nicht einmal, ob $L \subsetneq NP$ oder $L = NP$.
    \end{itemize}

    Es gibt noch leichtere Probleme als polynomielle. Praktisch relevante Probleme in L sind z.B. Erreichbarkeit in ungerichteten Graphen und die Zwei-Färbbarkeit von Graphen.
    Probleme, welche noch leichter zu lösen sind, sind ggf. nicht mehr durch Härte und Vollständigkeit bezeichenbar und bedürfen ggf. andererer Berechnungsmodelle. \vl{TIL 12}

    \begin{description}
        \item[Robustheit von Zeitklassen] Setzt sich aus zwei Erkenntnissen zusammen:
            \begin{itemize}
                \item Konstante Faktoren haben keinen Einfluss auf die Probleme, die eine zeitbeschränkte Mehrband-TM lösen kann, sofern mindestens lineare Zeit erlaubt ist (Linear Speedup Theorem). Sofern nicht einmal lineare Zeit zur Verfügung stände, könnte die TM nicht einmal die Eingabe lesen!
                \item Die Anzahl der Bänder hat lediglich einen polynomiellen (quadratischen) Einfluss auf die Probleme, die eine zeitbeschränkte TM lösen kann.
            \end{itemize}
        \item[Robustheit von Speicherklassen] Weder konstante Faktoren, noch die Anzahl der Bänder haben Einfluss auf die Probleme, welche eine speicherbeschränkte TM lösen kann.
    \end{description}

    \section{Prädikatenlogik}
Die Prädikatenlogik erweitert die Aussagenlogik. Neben den neuen Mengen der Variablen \f{V}, der Konstanten \f{C}, der Funktionen \f{F} und Prädikate \f{P} kommen der Allquantor $\forall$ und der Existenzquantor $\exists$ hinzu. Semantisch nutzen wir Interpretationen, um für Formeln Modelle zu finden (d.h. Variablenbelegungen zu finden, für welche die Formel nach wahr ausgewertet wird). \\
Formeln können mit Hilfe von syntaktischen Umformungen umgeformt und vereinfacht werden. Dazu nutzen wir Algorithmen zum logischen Schließen, z.b. die Unifikation und Resolution. \vl{TIL 13}

\subsection{Syntax}
    Im Gegensatz zu der unendlichen Menge von Atomen in der Aussagenlogik gibt es in der Prädikatenlogik die vier betrachteten Mengen V, C, F und P. Diese Mengen sind abzählbar unendlich und die Elemente disjunkt. Formeln sind, ausgenommen genannter Ausnahmen, eindeutig zu klammern. Die Mengen V, C und F arbeiten mit beliebigen Werten, die Prädikate hingegen werten bei Interpretation immer nach $true$ oder $false$ aus. \vl{TIL 13}
    \begin{description}
        \item[Variablen] Die Menge \f{V}, bestehend aus $x, y, z\dots$. \\
            Variablen können frei oder gebunden vorkommen (oder bei mehrfachem Auftreten einer Variable in einer Formel auch beides).

            \f{Freie} Variablen sind durch keinen Quantor gebunden. \\
            \f{Gebundene} Variablen befinden sich innerhalb des „Scope“ eines Quantors.

            Beispiel: in der Formel $p(x) \land \exists x.q(x)$ kommt $x$ sowohl frei ($p(x)$) als auch gebunden ($q(x)$) vor.

        \item[Konstanten] Die Menge \f{C}, bestehend aus $a,b,c,\dots$

        \item[Funktionen] Die Menge \f{F}, bestehend aus $f, g, h, \dots$. Stelligkeit (Arität) $\geq 0$.

        \item[Prädiktensymbole] Die Menge \f{P}, bestehend aus $p,q,r,\dots$. Stelligkeit $\geq 0$. Bei nullstelligen Prädiktensymbolen lassen wir die leeren Klammern weg.

        \item[Quantoren] Der Allquantor $\forall$ beschreibt, dass die betreffende Formel für alle möglichen Interpretationen der Variable gelten muss. Der Existenzquantor $\exists$ beschreibt, dass es mindestens eine gültige Interpretation der Variable geben muss. Wenn ein Quantor vor einer Formel mehrere Variablen betrifft, schreiben wir diese als Liste ($\forall x,y.F$ statt $\forall x.\forall y.F$).

        \item[Atom] Ein prädikatenlogisches Atom ist ein Ausdruck $p(t_{1},\dots,t_{n})$ für ein $n$-stelliges Prädiktensymbol $p \in \mathbf{P}$ und \f{Terme} $t_{1},\dots,t_{n}$. Hierbei gilt entweder $t_{1},\dots,t_{n} \in \mathbf{V} \cup \mathbf{C}$ oder $t_{n} = f(t_{1},\dots,t_{i})$ mit $f \in F$ $i$-stelliges Funktionssymbol und $t_{1},\dots,t_{i}$ wieder Terme.

        \item[Formel] Jedes Atom ist eine Formel. Wenn nun $x\in\f{V}$ und $F$ und $G$ Formeln, dann sind auch $\neg F$, $(F\land G)$, $(F\lor G)$, $(F\to G)$, $(F\leftrightarrow G)$, $\exists x.F$ und $\forall x.F$ Formeln. Die äußersten Klammern von Formeln dürfen weggelassen werden. Klammern innerhalb von mehrfachen Konjunktionen oder Disjunktionen dürfen weggelassen werden. Hat eine Formel keine freie Variablen ist sie \f{geschlossen} und wird \f{Satz} genannt, ansonsten ist sie eine \f{offene} Formel.

        \item[Teilformel] Teilformeln einer Formel sind alle Teilausdrücke einer Formel, welche selbst Formeln sind.
    \end{description}


\subsection{Semantik}
    Der Wahrheitswert von Formeln ergibt sich aus den Wahrheitswerten der Atome in dieser Formel. \vl{TIL 13}
    \begin{description}
        \item[Interpretation] Interpretation $\INT$  ist ein Paar $\langle\Delta^{\INT}, \cdot^{\INT}\rangle$.\\
            Die nichtleere Menge $\Delta^{\INT}$ wird auch \f{Domäne} genannt. \\
            Die Funktion $\cdot^{\INT}$ heißt \f{Interpretationsfunktion}. Diese bildet
            \begin{itemize}
                \item jede Konstante $a \in \f{C}$ auf ein Element $a^{\INT} \in \Delta^{\INT}$,
                \item jedes $n$-stellige Funktionssymbol $f \in F$ auf eine $n$-stellige Funktion $f^{\INT}: (\Delta^{\INT})^{n} \to \Delta^{\INT}$ und
                \item jedes $n$-stellige Prädiktensymbol $p \in \f{P}$ auf eine Relation $p^{\INT} \in (\Delta^{\INT})^{n}$ ab.
            \end{itemize}

        \item[Zuweisung] Zuweisung $\ZUW$ für eine Interpretation $\INT$ ist eine Funktion $\ZUW: \f{V} \to \Delta^{\INT}$, sie bildet also Variablen auf Elemente der Domäne ab.
            Bei $x \in \f{V}$ und $\delta \in \Delta^{\INT}$ schreiben wir für die Zuweisung von $x$ auf $\delta$ und für alle $y \neq x$ auf $\ZUW (y)$: $\ZUW[x \mapsto \delta]$.{}

        \item[Wahrheitsbestimmung] Die Wahrheitsbestimmung von Atomen und Formeln unter einer Interpretation und einer Zuweisung werden rekursiv aufgelöst.
            \begin{itemize}
                \item Für Konstanten $c$ benötigen wir nur die Interpretation: $c^{\INT,\ZUW} = c^{\INT}$
                \item Für Variablen $x$ benötigen wir nur die Zuweisung: $x^{\INT,\ZUW} = \ZUW(x)$
                \item Für einen Funktionsterm $t = f(t_{1},\dots,t_{n})$ definieren wir: $t^{\INT, \ZUW} = f^{\INT}(t_{1}^{\INT, \ZUW}, \dots, t_{n}^{\INT, \ZUW})$
                \item Für Prädikate/Atome $p(t_{1},\dots,t_{n})$ setzen wir nun rekursiv: \\
                    $p(t_{1},\dots,t_{n})^{\INT,\ZUW} = 1$ wenn $\langle t_{1}^{\INT,\ZUW},\dots,t_{n}^{\INT,\ZUW} \rangle \in p^{\INT}$ bzw. \\
                    $p(t_{1},\dots,t_{n})^{\INT,\ZUW} = 0$ wenn $\langle t_{1}^{\INT,\ZUW},\dots,t_{n}^{\INT,\ZUW} \rangle \notin p^{\INT}$
            \end{itemize}

            Für eine Formel gilt nun: \\
            eine Interpretation $\INT$ und eine Zuweisung $\ZUW$ \f{erfüllen} eine Formel $F$, geschrieben „$\INT, \ZUW \models F$“, wenn die Rekursion mit Atomen, Operationen und Quantoren zu Wahr auflöst.
    \end{description}


\subsection{Semantische Grundbegriffe}
    Wir wollen in der Prädikatenlogik wenn möglich nur mit Sätzen arbeiten, d.h. mit geschlossenen Formeln ohne ungebundene Variablen. \vl{TIL 14}
    \begin{description}
        \item[Modelltheorie] Wir unterscheiden grob zwischen der Prädikatenlogik mit und ohne offenen Formeln. Bei der Prädikatenlogik mit Sätzen können wir auf Zuweisungen verzichten. Formeln sind Behauptungen, die wahr oder falsch sein können. Modelle sind mögliche Welten (prädikatenlogische Interpretationen und ggf. Zuweisungen), in denen manche Behauptungen gelten und andere nicht. (\f{Intuition})

        \item[Typen von Formeln]
            Siehe hierzu auch die Graphen „Modelle $\models$ Formeln“ in \vl{TIL 14}
            \begin{itemize}
                \item allgemeingültig (tautologisch): Eine Formel, die in allen Modellen wahr ist
                \item widersprüchlich (inkonsistent): Eine Formel, die in keinem Modell wahr ist
                \item erfüllbar: Eine Formel, die in einem Modell wahr ist
                \item widerlegbar: Eine Formel, die in einem Modell falsch ist
            \end{itemize}

        \item[Logisches Schließen] Bei der Analyse von Modellen für Formeln und andersherum können in Wechselwirkung Konsequenzen hergestellt werden.
            \begin{enumerate}
                \item Wenn $\INT$ die Formel $F$ erfüllt, also $\INT \models F$, dann ist $\INT$ ein Modell für $F$.
                \item $\INT$ kann mehrere Formeln erfüllen, d.h. sie kann Modell für eine Formelmenge $\FM$ sein, wenn $\INT$ alle Formen in $\FM$ erfüllt.

                \item Eine Formel $F$ ist nun eine \f{logische Konsequenz} aus einer Formel bzw. Formelmenge $G$, d.h. $G \models F$, wenn jedes Modell $\INT$ von $G$ auch ein Modell von $F$ ist, d.h. $\INT \models G \implies \INT \models F$. \\
                Sonderfall: Ist $F$ eine Tautologie, dann schreiben wir nur $\models F$. \\

                Beispiel 1: Gegeben sind vier Modelle $\INT_{i}$ und vier Formeln $F_{j}$. $\INT_{2}$ und $\INT_{3}$ sind alle erfüllenden Modelle für $F_{3}$. $\INT_{2}$ und $\INT_{3}$ sind aber u.a. auch Modelle für $F_{2}$. Das bedeutet, wenn $F_{3}$ erfüllt ist, ist auch immer $F_{2}$ erfüllt. Es gilt $F_{3} \models F_{2}$. \\

                Beispiel 2: Im Beispiel der Logelei „Wir sind alle vom gleichen Typ“ haben wir fünf Formeln gegeben. Drei davon ergeben sich aus den gegebenen Aussagen („gegebene Theorie“) und die anderen beiden sind Allquantor-Behauptungen für „alle sagen die Wahrheit“ bzw. „alle lügen.“. Wir können anhand der Modelle „LL“, „WL“ und „WW“ und der Theorie Konsequenzen erstellen und somit über das Modell „WW“ die Behauptung „$\forall x.W(x)$“ als logische Konsequenz für unsere Theorie identifizieren.

                \item Zwei Formelmengen $F$ und $G$ können auch semantisch äquivalent sein, d.h. $F \equiv G$, wenn sie genau die gleichen Modelle haben ($\INT \models F$ gdw. $\INT \models G$ für alle Modelle $\INT$).
            \end{enumerate}
        \item[Semantische Äquivalenz] Eine Äquivalenzrelation $\equiv$ ist reflexiv, symmetrisch und transitiv. Alle Tautologien sind semantisch äquivalent. Alle unerfüllbaren Formeln sind semantisch äquivalent. \\ Äquivalenz $F \equiv G$ gdw. $F\models G$ und $G \models F$.
        \item[Problem logischen Schließens in der Prädikatenlogik] Die zwei Fragen „Model checking“ (Überprüfung eines Modells auf Erfüllung einer Formel) und „Logische Folgerung (Entailment)“ (Überprüfung ob zwei Formeln oder Formelmengen eine logische Konsequenz sind) sind in der Prädikatenlogik schwerer zu lösen als in der Aussagenlogik.
        \item[Monotonie und Tautologie] Aus der Definition von $\models$ folgt die Monotonie: je mehr Sätze in einer logischen Theorie gegeben sind, desto weniger Modelle können die gesamte Theorie erfüllen und desto mehr Schlussfolgerungen kann man aus der logischen Theorie ziehen. D.h. mehr Annahmen führen zu mehr Schlussfolgerungen. Extremfälle sind hierbei Tautologien (sind in jedem Modell wahr und daher logische Konsequenz jeder Theorie) und unerfüllbare Formeln (sind in keinem Modell wahr und haben daher alle anderen Sätze als Konsequenz).

        %\item[Beziehung zur Aussagenlogik] In der Semantik wird nur die Wertzuweisung ersetzt. Hier gibt es nun Interpretationen und Zuweisungen.

        \item[Gleichheit] Es gibt ein spezielles Gleichheitsprädikat $\approx$. In Interpretationen $\INT$ gilt $\approx^{\INT} = \{\langle\delta,\delta\rangle|\delta\in\Delta^{\INT}\}$.
        Dies kann z.B. zum Erzwingen von gleicher Interpretation von Konstanten verwendet werden. Auch gibt es $\not\approx$, Definition $\forall x,y.(x\not\approx y \leftrightarrow \neg x\approx y)$. Man kann aber mit Hilfe anderer Definitionen der Prädikatenlogik sowohl Gleichheit als auch Ungleichheit einsparen. \vl{TIL 14} \vl{TIL 15}
    \end{description}


\subsection{Prädikatenlogik als Universalsprache}
    Die Entwicklung der Logik hat ein zentrales Motiv: Logik als eine universelle, präzise Sprache. Die Entwicklung begann bei Aristoteles als Grundlage der philosophischen Argumentation, ging in Leibniz Sinne in Richtung „rechnen“ und wurde von Hilbert und Russell schließlich zusammen mit der Mathematik formalisiert. Wenn nun die Mathematik in logischen Formeln formuliert wird, wird logisches Schließen zur Kernaufgabe der Mathematik. Eine zentrale Frage des Schließens ist hierbei die Überprüfung auf Erfüllbarkeit einer Formel bzw. einer Formelmenge. \vl{TIL 15}
    \begin{description}
        \item[Strukturelle Induktion] Diese Induktion kann man über jede induktiv definierte syntaktische Struktur durchführen (z.B. Formeln, Terme, Programme,\dots).
            \begin{itemize}
                \item In der „klassischen Induktion“ wird eine Eigenschaft $E$ untersucht, wobei (1) „0 hat $E$“ geprüft und darauf aufbauend (2) für alle $n>0$ im Falle von „$n-1$ hat $E$“ geprüft wird.
                \item In der \f{strukturellen Induktion auf Formeln} prüfen wir nun ob (1) alle atomaren Formeln $E$ haben und (2) alle nicht-atomaren Formeln $F$ ebenfalls $E$ haben, wenn alle ihre echten Teilformeln $E$ haben.
            \end{itemize}

        Im Beispiel „Induktion auf der Insel der Wahrheitssager und Lügner. Ein Einwohner verkündet: 'Was ich jetzt sage, das habe ich schon einmal gesagt.' Welchen Typ hat er?“ muss der Einwohner ein Lügner sein, da er mindestens beim ersten Mal lügt.
    \end{description}


\subsection{Unentscheidbarkeit des logischen Schließens}
    Erinnerung: $F$ ist logische Konsequenz von $G$ ($F\models G$), wenn alle Modelle von $F$ auch Modelle von $G$ sind. (1) Es ist nicht offensichtlich, wie man das überprüfen sollte, denn es gibt unendliche viele Modelle. (2) Ebenso schwer erscheinen die gleichwertigen Probleme der Erfüllbarkeit und Allgemeingültigkeit. \\

    Intuition: prädikatenlogisches Schließen ist unentscheidbar. Beweis durch Reduktion eines bekannten unentscheidbaren Problems, z.B. Halteproblem, PCP, Äquivalenz kontextfreier Sprachen u.a. \\

    Der Beweis in der Vorlesung zeigt die Reduktion vom CFG-Schnittproblem. Hierfür werden Wörter $\omega$ aus der Modellmenge (Modellstruktur) $\INT$ als Ketten von binären Relationen kodiert und untersucht, ob das Wort $\omega$ in der Schnittmenge zweier kontextfreier Grammatiken $G_{1}$ und $G_{2}$ vorkommt. \\
    Beispiel: wir haben auf der Insel z.B. das Modell mit Kombination „LLWWW“ (drei sagen die Wahrheit, zwei lügen), und wir wollen wissen ob $F \models G$. Wir kodieren die erfüllenden Modelle der Formeln $F$ und $G$ wie o.g. und erhalten $G_{1}$ und $G_{2}$. Nach Kodierung müssten also in beiden Grammatiken die Übergänge $\langle L_{1},L_{2} \rangle, \langle L_{2},W_{1} \rangle, \langle W_{1},W_{2} \rangle, \langle W_{2},W_{3} \rangle$ vorkommen. Ist dies der Fall, dann erfüllt dieses Modell beide Formeln.  (\textit{Vergleich und Notation nicht nach VL!}) \\
    Zusammenfassend lassen sich demnach logische Konsequenzen auf diese Probleme reduzieren und da CFG unentscheidbar gilt auch: Logisches Schließen (Erfüllbarkeit, Allgemeingültigkeit, logische Konsequenz) in der Prädikatenlogik ist unentscheidbar. \vl{TIL 15}

\subsection{Gödel}
    Gödelscher Vollständigkeitssatz und Unvollständigkeitssätze. \vl{TIL 15} \vl{TIL 21}
    \begin{description}
        \item[Gödelscher Vollständigkeitssatz] „Es gibt ein konsistentes Verfahren, das alle Konsequenzen einer prädikatenlogischen Theorie effektiv beweisen kann.“ (1) Alle wahren Sätze können endlich bewiesen werden. (2) Prädikatenlogisches Schließen ist semi-entscheidbar.
        \item[1. Gödelscher Unvollständigkeitssatz] „Es gibt kein konsistentes Verfahren, das alle Konsequenzen der elementaren Arithmetik effektiv beweisen kann.“ (1) Für jedes Verfahren gibt es Sätze über elementare arithmetische Zusammenhänge, die weder bewiesen noch widerlegt werden können. (2) Die Wahrheit elementarer arithmetischer Zusammenhänge ist nicht semi-entscheidbar.
        %~ \item[2. Gödelscher Unvollständigkeitssatz]
    \end{description}

\subsection{Syntaktische Umformungen}
    \begin{description}
        \item[Äquivalenzen mit Quantoren] Es gelten die folgenen Beziehungen: \vl{TIL 16}
            \begin{itemize}
                \item Negation von Quantoren: $\neg\exists.F \equiv \forall x.\neg F$ und $\neg\forall x.F \equiv \exists x.\neg F$
                \item Kommutativität: $\exists x. \exists y.F \equiv \exists y. \exists x.F$, selbiges für $\forall${}
                \item Distributivität: $\exists x.(F \lor G) \equiv (\exists x.F \lor \exists x.G)$, selbiges für $\forall / \land$
            \end{itemize}
            Wichtig: andere Kombinationen funktionieren \f{nicht} ohne dass die Semantik verändert wird.

        \item[Negationsnormalform (NNF)] Enthält nur Quantoren und die Junktoren $\land$, $\lor$ und $\neg$. \\
            Der Negator $\neg$ befindet sich nur noch direkt vor Atomen (Literalen). \\
            Zum Umformeln beginnen wir zuerst mit der Ersetzung von $\to$ und $\leftrightarrow$: \\
            $(F \to G) \equiv (\neg F \lor G)$ und $(F \leftrightarrow G) \equiv (\neg F \lor G) \land (\neg G \lor F)$. \\
            Folgend wird NNF($F$) rekursiv umgeformelt. Hierbei können z.B. Quantoren, die in ihrem Scope keine freie Variable binden, entfernt werden.

        \item[Bereinigte Formel, Variablenumbenennung] Gebundene und freie Variablen in Formeln können umbenannt werden, so lange die neue Bezeichnung nicht bereits in der Formel vorkommt. Eine Formel ist \f{bereinigt}, wenn in ihr (1) keine Variable sowohl ungebunden als auch gebunden vorkommt und (2) keine Variable von mehr als einem Quantor gebunden wird. Beispiel: \\
            Die Formel $\forall y.p(x,y) \to \exists x.(r(y,x) \land \forall y.q(x,y))$ wird zu $\forall y.p(x,y) \to \exists z.(r(y,z) \land \forall v.q(z,v))$.

        \item[Pränexform] In der Pränexform stehen alle Quantoren am Anfang einer Formel, d.h. $Q_{1}x_{1}.Q_{2}x_{2}.\dots Q_{n}x_{n}.F$, wobei $Q_{n}x_{n}$ Quantor mit Variable. Die Umformung einer Formel in Pränexform geschieht nach NNF und Bereinigung, da wir dann ohne Komplikationen alle Quantoren aus der Formel herausziehen können (da jede Variable nur an maximal einem Quantor gebunden ist). \vl{TIL 17}

        \item[Skolemisierung] Die Skolemisierung baut auf der Pränexform auf. Nach erfolgreicher Umformung sind alle Existenzquantoren eliminiert und das Vorkommen der entsprechenden Variable durch einen Funktionsterm (Skolemterm) ersetzt. \\
            Sei $\forall x_{1}\dots \forall x_{n}. \exists y.F$ eine Formel in Pränexform.
            Dann erstellen wir die neue Formel $\forall x_{1} \dots \forall x_{n}.F'$ mit $F' = F\{y \mapsto f(x_{1},\dots,x_{n})\}$. Die Variable $y$ wird also durch den Skolemterm $f$, eine $n$-stellige Skolemfunktion mit bisher unverwendetem Bezeichner, ersetzt.
            Die Parameter der Funktion sind die Variablen der Allquantoren vor dem eliminierten Existenzquantor.
            Mehrere Existenzquantoren werden von links nach rechts aufgelöst.

            Beispiel: $\forall x. \exists y. \forall z. \exists v.p(x,y,z,v)$ $\longrightarrow$
            $\forall x. \forall z. \exists v.p(x, f(x), z, v)$ $\longrightarrow$
            $\forall x. \forall z.p(x, f(x), z, g(x,z))$

            Skolemisierung kann die Semantik einer Formel verändern, jedoch bleibt die Erfüllbarkeit erhalten.

        \item[Konjunktive Normalform (KNF)] Eine Formel ist in konjunktiver Normalform, wenn sie eine Konjunktion von Diskunktionen von Literalen ist:
            $(L_{1,1} \lor L_{1,2} \lor \dots) \land \dots \land (L_{n,1}, L_{n,2}, \dots)$. \\
            Zum Umformeln muss eine Formel (1) bereinigt, (2) in NNF umgeformt, (3) in Pränexform gebracht und (4) skolemisiert werden. \\
            Zum Abschluss wird noch die Ersetzung $F \lor (G \land H) \mapsto (F \lor G) \land (F \lor H)$ angewandt.

        \item[Klauselform] Hierfür wird die KNF nochmals vereinfacht.
            \begin{itemize}
                \item Allquantoren werden weggelassen.
                \item Klauseln werden als Mengen von Literalen geschrieben.
                \item Konjunktionen von Klauseln werden als Mengen von Mengen von Literalen geschrieben.
            \end{itemize}

    \end{description}

\subsection{Algorithmen zum logischen Schließen}
    \begin{description}
        \item[Substitution] In der Substitution werden freie Variablen $x \in V$ durch Terme $t \in T$ ersetzt.
            Eine Substitution wird durch $\sigma$ o.ä. definiert, z.B. $\sigma = \{x_{1} \mapsto t_{1}, \dots \}$. Wird diese Substitution dann auf eine Formel $A$ angewandt, d.h. $A\sigma$, nennt man dies \f{Instanz} von $A$ unter $\sigma$. Man kann mehrere Substitutionen hintereinander ausführen. Dann gilt $A(\sigma\circ\theta) = (A\sigma)\theta$.{}

        \item[Unifikation] Ein Unifikationsproblem ist eine endliche Menge von Gleichungen der Form \\ $G = \{s_{1} \doteq t_{1}, \dots , s_{n} \doteq t_{n} \}$.
            Eine Substitution $\sigma$ ist ein Unifikator für $G$ falls $s_{i}\sigma = t_{i}\sigma$ für alle $i \in \{ 1, \dots , n \}$ gilt. \vl{TIL 18}

            Es kann mehrere Substitutionen geben, die diese Anforderung erfüllen. Dann kann durch Vergleich ein \f{allgemeinster Unifikator} gefunden werden.

            Eine Substitution $\sigma$ ist \f{allgemeiner} als eine Substitution $\theta$, in Symbolen $\sigma \preceq \theta$, wenn es eine Substitution $\lambda$ gibt, so dass $\sigma \circ \lambda = \theta$. Der allgemeinste Unifikator für ein Unifikationsproblem $G$ ist ein Unifikator $\sigma$ für $G$, so dass $\sigma \circ \theta$ für alle Unifikatoren $\theta$ für $G$. Die englische Bezeichnung des allgemeinsten Unifikators ist \f{most general unifier (mgu)}.

            Ein Unifikationsproblem $G = \{x_{1} \doteq t_{1}, \dots , x_{n} \doteq t_{n} \}$ ist in \f{gelöster Form}, wenn $x_{1}, \dots, x_{n}$ paarweise verschiedene Variablen sind, die nicht in den Termen $t_{1}, \dots, t_{n}$ vorkommen. In diesem Fall definieren wir eine Substitution $\sigma_{G} := \{ x_{1} \mapsto t_{1}, \dots , x_{n}\mapsto t_{n} \}$. Dann ist $\sigma_{G}$ ein allgemeinster Unifikator für $G$.

            \f{Algorithmus:}
            \begin{itemize}
                \item Löschen (überflüssige $\doteq$ löschen, z.B. $\{ f(x) \doteq f(x) \}$)
                \item Zerlegung (Parameter gleicher Funktionen auflösen, \\
                    z.B. $\{g(a, f(x)) \doteq g(b, f(x))\}$ wird $\{a \doteq b, f(x) \doteq f(x) \}$)
                \item Orientierung (Variablen auf die linke Seite)
                \item Eliminierung (gegebene Variablen durch Wert ersetzen, \\
                    z.B. $\{ x \doteq f(a), g(x) \doteq g(y) \}$ wird $\{ x \doteq f(a), g(f(a)) \doteq g(y)\}$).
            \end{itemize}

        \item[Resolution]
            Mit dem Resolutionsalgorithmus versuchen wir aus einer gegebenen Klauselmenge (d.h. eine Formel in Klauselform) eine leere Klausel zu erzeugen (abzuleiten). Diese leere Klausel wäre eine unerfüllbare Behauptung, d.h. sobald wir eine solche leere Klausel finden haben wir die Unerfüllbarkeit der Formel bewiesen. Die Erfüllbarkeit ist hierbei eine „zentrale Frage des Schließens“.

            \f{Algorithmus:} gegeben eine Klauselmenge, welche nummeriert angeordnet sind.
            Nachfolgend nehmen wir immer zwei Klauseln, welche in Kombination wahre und falsche Aussagen resolvieren. Hierbei müssen die Variablen der neu erzeugten Klauseln umbenannt werden, es entstehen \f{Varianten}. \vl{TIL 15} \vl{TIL 16} \vl{TIL 19}

            Beispiel: gegeben sind (1) und (2), neu erzeugt wird (3) \\
            (1) $\{ W(x_{1}), L(x_{2}) \}$ \hspace{1cm}
            (2) $\{ \neg W(a)\}$ \hspace{1cm}
            (3) $\{ L(x_{2}')\}$ \textcolor{blue}{(1) + (2), $\{ x_{1} \mapsto a \}$} \\
            %~ \textit{(Dieses Beispiel zeigt nicht die vollständige Resolution mit einer leeren Klausel)}

        \item[Herbrand] Herbrand-Universum, Herbrandinterpretationen und Herbrandmodelle. \vl{TIL 19} \\
            Das Herbranduniversum ist eine Erzeugung einer „Semantik aus Syntax“, einer Konstruktion von Modellen direkt aus Formeln.

            Sei $a$ eine beliebige Konstante. Das Herbranduniversum $\HU$ für eine Formel $F$ ist die Menge aller variablenfreien Terme, die man mit Konstanten und Funktionssymbolen in $F$ und der zusätzlichen Konstante $a$ bilden kann:
            \begin{itemize}
                \item $a \in \HU$
                \item $c \in \HU$ für jede Konstante aus F
                \item $f(t_{1}, \dots, t_{n}) \in \HU$ für jedes $n$-stellige Funktionssymbol aus $F$ und alle Terme $t_{1}, \dots, t_{n} \in \HU$
            \end{itemize}
            Anmerkung: Das Herbrand-Universum ist immer abzählbar, manchmal endlich und niemals leer. \\
            Beispiel: Für die Formel $F = p(f(x), y, g(z))$ ergibt sich $\HU = \{a, f(a), g(a), f(f(a)), f(g(a)), \dots \}$.

\newpage
            Eine Herbrandinterpretation für eine Formel $F$ ist eine Interpretation $\INT$ für die gilt:
            \begin{itemize}
                \item $\Delta^{\INT} = \HU$ ist das Herbrand-Universum von $F$
                \item Für jeden Term $t \in \HU$ gilt $t^{\INT} = t$
            \end{itemize}
            D.h. Prädikate können wie in einer üblichen Interpretation unterschiedliche Werte erhalten.

            $\INT$ ist ein Herbrandmodell für $F$ wenn zudem gilt $\INT \models F$.

            Die \f{Herbrand-Expansion} $\HE(F)$ einer Formel $F = \forall x_{1}, \dots, x_{n}.G$ in Skolemform ist die Menge:
                $\HE(F) := \{ G\{x_{1} \mapsto t_{1}, \dots, x_{n} \mapsto t_{n} \}\ |\ t_{1},\dots,t_{n} \in \Delta_{F}\ \}$

            Satz von Gödel, Herbrand und Skolem: Eine Formel $F$ in Skolemform ist genau dann erfüllbar, wenn $\HE(F)$ aussagenlogisch erfüllbar ist.

        \item[Lifting-Lemma] Seien $K_{1}$ und $K_{2}$ prädikatenlogische Klauseln mit Grundinstanzen $K_{1}' = K_{1} \sigma$ und $K_{2}' = K_{2} \sigma$. Wenn $R'$ eine (aussagenlogische) Resolvente von $K_{1}'$ und $K_{2}'$ ist, dann gibt es eine prädikatenlogische Resolvente $R$, welche $R'$ als Grundinstanz hat. \vl{TIL 20}

        \item[Kompaktheit] Satz (Endlichkeitssatz, Kompaktheitssatz): Falls eine unendliche Menge prädikatenlogischer Sätze $\mathcal{T}$ eine logische Konsequenz $F$ hat, so ist $F$ auch Konsequenz einer endlichen Teilmenge von $\mathcal{T}$.

        \item[Endliche Modelle] Satz von \f{Löwenheim und Skolem}: „Jede erfüllbare prädiktenlogische Formel hat ein abzählbares Modell (d.h. eines mit abzählbarer Domäne).“ (aber: nicht jede Formel hat ein endliches Modell!). \vl{TIL 20}

            Beispiel anhand von relationalen Datenbanken in \vl{TIL 20}

        \item[Model Checking] Das Auswertungsproblem (Model Checking) der Prädikatenlogik lautet wie folgt: \\
            Gegeben: Eine Formel $Q$ mit freien Variablen $x_{1}, \dots, x_{n}$; eine endliche Interpretation $\INT$; Elemente $\delta_{1}, \dots, \delta_{n} \in \Delta^{\INT}$. Frage: Gilt $\INT$, $\{x_{1} \mapsto \delta_{1}, \dots, x_{n} \mapsto \delta_{n} \} \models Q$?

    \end{description}


    \newpage
    \section{Übungen}
Visuelle Hilfen, die in einigen Lösungen verwendet werden, versuche ich schriftlich zu beschreiben. Hier hilft es ggf., die gegebenen Lösungen nochmals selber aufzuschreiben um die logischen Schritte besser nachvollziehen zu können.

\input{includes/ueb/01}
\input{includes/ueb/02}
\input{includes/ueb/03}
\newpage

\subsection*{Übung 4 (Berechenbarkeitstheorie)}
\subsubsection*{Aufgabe 1}
    Besitzen folgende Instanzen $P_{i}$ des Postschen Korrespondenzproblems Lösungen oder nicht?
    \begin{enumerate}
        \item Ja, einfach zu zeigen.
        \item Nein, denn der erste Stein ist der einzige, mit dem begonnen werden kann. Folgend passt nur der dritte Stein und nach diesem ebenfalls immer nur der dritte. Das untere Wort ist demnach immer länger als das obere.
        \item Ja, hat Lösung mit 66 Steinen.
    \end{enumerate}


\subsubsection*{Aufgabe 2}
    Zeigen Sie, dass das Postsche Korrespondenzproblem über einem einelementigen Alphabet entscheidbar ist.

    \LOES Dies lässt sich mit Hilfe eines Algorithmus lösen, welcher die Länge der Wortpaare untersucht.
    Sei $P = (a^{u_{1}}, a^{v_{1}}), \dots, (a^{u_{n}}, a^{v_{n}})$ über $\Sigma = \{ a \}$. Wir schreiben nun $(u_{i}, v_{i})$ statt $(a^{u_{i}}, a^{v_{i}})$, betrachten also nur die jeweilige Länge. Fallunterscheidung:
    \begin{itemize}
        \item 1. Fall: Es gibt ein $u_{i} = v_{i}$. Dann ist Paar $i$ die Lösung.
        \item 2. Fall: Alle $i$ sind derart, dass $u_{i} < v_{i}$ bzw. $u_{i} > v_{i}$, d.h. alle oberen bzw. unteren Wörter sind länger als das jeweils andere. Dann ist $P$ unlösbar.
        \item 3. Fall: Es gibt $i, j$ mit $u_{i} > v_{i}$ und $u_{j} < v_{j}$. Eine Lösung hat dann die Form $(\overbrace{i,i,\dots}^\text{k-mal},\overbrace{j,j,\dots}^\text{l-mal})$ (sofern Lösung existiert). Dann muss gelten $k \cdot u_{i} + l \cdot u_{j} = k \cdot v_{i} + l \cdot v_{j}$, also $k \cdot (u_{i} - v_{i}) = l \cdot (v_{j} - u_{j})$. Wähle $l = (u_{i} - v_{i})$, $k = (v_{j} - u_{j})$. Dann ist $(k \cdot i, l \cdot j)$ tatsächlich eine Lösung.
    \end{itemize}
    In jedem Fall ist also entscheidbar, ob es eine Lösung gibt. Damit ist die Aussage gezeigt.


\subsubsection*{Aufgabe 3}
    Zeigen Sie, dass folgendes Problem unentscheidbar ist: gegeben eine Turing-Maschine $M$ und ein $k \in \N$, kann die Sprache $L(M)$ durch eine Turing-Maschine mit höchstens $k$ Zuständen erkannt werden?
    Zeigen Sie dazu, dass für $k = 1$ die Menge $T_{k} := \{\ enc(M)\ |\ L(M)\ \text{wird von einer TM mit höchstens}\ k\ \text{Zuständen erkannt} \}$ nicht entscheidbar ist. Warum zeigt dies die ursprüngliche Behauptung? \\

    \LOES Wir setzen hier den Satz von Rice über die Unentscheidbarkeit von Eigenschaften von Sprachen an. (\textit{Wichtig! Nicht Eigenschaften von Maschinen!}) \\
    Wir betrachten den Fall $T_{1}$. $T_{1}$ ist nach Satz von Rice unentscheidbar. \\
    Was ist in diesem Falle die Eigenschaft $E$? Definition für $L \subset \SIGS$. \\
    \boxed{L$ erfüllt $E \Longleftrightarrow$ es gibt eine TM $N$ mit einem Zustand, so dass $L(N) = L}. Dann ist $E$ Eigenschaft von Sprachen. Bemerkung: ist $L$ nicht semi-entscheidbar (benötigt für Satz von Rice!), dann gibt es keine TM $N$  mit $L = L(M)$. Also erfüllt $L$ die Eigenschaft $E$ nicht. \\
    $E$ ist nicht-trivial: $L = \emptyset$ funktioniert, TM hat keinen Endzustand. $L = \{ aa \}$ jedoch funktioniert nicht, da mit nur einem Zustand nicht gezählt werden kann.
    Also ist nach Satz von Rice die Maschine $T_{1}$ unentscheidbar. \\
    Da das Problem bereits für $k=1$ unentscheidbar ist, ist es auch für beliebige $k$ unentscheidbar. \\
    Der Sonderfall $k=0$ führt zu $T_{0} = \emptyset$. Ist $\emptyset$ entscheidbar? Ja, die TM lehnt einfach immer ab. Da dieser Fall jedoch trivial ist, fällt er nicht unter den Satz von Rice.

\newpage
\subsubsection*{Aufgabe 4}
\label{U4-4}
    Zeigen Sie, dass weder das Äquivalenzproblem $\prspec{"aquiv}$ für Turing-Maschinen noch dessen Komplement $\overline{\prspec{"aquiv}}$ semi-entscheidbar ist, wobei
    \begin{itemize}
        \item $\prspec{"aquiv} := \{ enc(M_{1}) \#\# enc(M_{2}) | L(M_{1}) = L(M_{2}) \}$
        \item $\overline{\prspec{"aquiv}} := \{ enc(M_{1}) \#\# enc(M_{2}) | L(M_{1}) \neq L(M_{2}) \}$
    \end{itemize}
    Zeigen Sie dazu, dass $\phalt \leq_{m} \prspec{"aquiv}$ und $\phalt \leq_{m} \overline{\prspec{"aquiv}}$ gilt. Weshalb zeigt dies die Aussage? \\

    \LOES Angenommen $\paq$ wäre semi-entscheidbar. Dann $\overline{\paq}$ co-semi-entscheidbar. \\ Da $\phalt \leq_{m} \overline{\paq}$ muss auch $\phalt$ co-semi-entscheidbar. Widerspruch! Also ist $\overline{\paq}$ nicht semi-entscheidbar. \\
    Selbige Heransgehensweise gilt für die entsprechenden Komplemente. \\

    Wir zeigen $\phalt \leq_{m} \paq$. Dafür geben wir eine berechenbare Funktion $f: \SIGS \to \SIGS$ an, so dass \boxed{enc(M) \#\# enc(w) \in \phalt \Leftrightarrow f(enc(M) \#\# enc(w)) \in \paq} für $M$ Turingmaschine und $w$ Eingabe.
    Dafür müssten wir zwei TM $M_{1}$ und $M_{2}$ finden, so dass gilt: \boxed{M$ hält auf $w \Leftrightarrow L(M_{1}) = L(M_{2})}.
    Seien also $M$ und $w$ wie oben. \\

    Dafür setzen wir $M_{1} =$ bei Eingabe $w$
    \begin{itemize}
        \item akzeptiere ($L(M_{1}) = \SIGS$)
    \end{itemize}
    Und $M_{2} = $ bei Eingabe $y$
    \begin{itemize}
        \item simuliere $M$ auf $w$ (nimmt $y$, codiert dies als $w$)
        \item akzeptiere (bedeutet $M$ hat gehalten. Andernfalls würde $M$ nicht halten)
    \end{itemize}
    Bedeutet: $L(M_{2})$ ist $\SIGS$, falls $M$ auf $w$ hält. Sonst $L(M_{2}) = \emptyset$. \\
    Dann gilt: \boxed{M$ hält auf $w \Leftrightarrow L(M_{2}) = \SIGS = L(M_{1})} \\
    Daher ist \boxed{f(enc(M) \#\# enc(w)) := enc(M_{1})\#\# enc(M_{2})} eine Reduktion von $\phalt$ auf $\paq$. \\
    Die Reduktion $\phalt \leq_{m} \overline{\paq}$ verläuft analog.

\subsection*{Übung 5 (Komplexitätstheorie)}
\subsubsection*{Aufgabe 1}
    Welche der folgenden Aussagen sind wahr? Begründen Sie Ihre Antwort.
    \begin{enumerate}
        \item Falls $P \neq \NP$ gilt, dann auch $P \cap \NP = \emptyset$.{}
        \item Es gibt Probleme, die NP-hart, aber nicht NP-vollständig sind.
        \item Polynomielle Reduzierbarkeit ist nicht transitiv.
        \item Ist $L_{2} \in P$ und $L_{1} \leq_{p} L_{2}$, dann ist auch $L_{1} \in P$.
        \item Ist $L_{1}$ eine NP-vollständige Sprache und gilt $L_{1} \leq_{p} L_{2}$, dann ist auch $L_{2}$ NP-vollständig.
        \item Ist $L_{2}$ eine NP-vollständige Sprache und gilt $L_{1} \leq_{p} L_{2}$, dann ist auch $L_{1}$ NP-vollständig.
    \end{enumerate}

    \LOES
    \begin{enumerate}
        \item Falsch. Es gilt $P \cap \NP = P \neq \emptyset$.
        \item Richtig. Jedes NP-Problem ist bspw. in polynomieller Zeit auf das Halteproblem $\phalt$ reduzierbar, aber $\phalt$ ist nicht in NP (da unentscheidbar).
        \item Falsch. Reduktion ist transitiv, die Komposition von polynomiell-zeitberechenbaren Funktionen ist wieder polynomiell-zeitberechenbar. Formell: $C \leq_{p} B \leq_{p} A \Rightarrow C \leq_{p} A$.
        \item Richtig. Ein Entscheidungsverfahren für $L_{1}$, welches in polynomieller Zeit läuft, reduziert zuerst die Eingabe $w$ auf eine Instanz $f(w)$ für $L_{2}$ und prüft dann, ob $f(w) \in L_{2}$.
        \item Falsch. $L_{2}$ muss nur NP-hart sein. Beispiel $\phalt$.
        \item Falsch. Beispiel $L = \emptyset$, $\emptyset \leq SAT$.
    \end{enumerate}


\newpage
\subsubsection*{Aufgabe 2}
    Zeigen Sie, dass das Wortproblem deterministischer endlicher Automaten in $L$ liegt: ist \\
    $\prspec{DFA} := \{\ enc(A) \#\# enc(w)\ |\ A\ \text{ist ein DFA, der $w$ akzeptiert} \}$, dann gilt $\prspec{DFA} \in L$. \\

    \LOES Die Klasse $L$ ist $LogSpace$, d.h. der Automat hat zusätzlich zur Eingabe logarithmisch viel Platz für seine Berechnung. Die Klasse $L$ ist somit die Klasse von Problemen, die mit einer konstanten Anzahl von Zählern und Zeigern gelöst werden können. \\

    Für die Simulation von $A$ auf $w$ brauchen wir
    \begin{itemize}
        \item einen Zeiger, der auf den aktuellen Zustand zeigt
        \item einen Zeiger in die Eingabe $w$
        \item 2-3 Hilfszähler
        \item 1-2 Zähler, um Eingabe zu überprüfen
    \end{itemize}
    Wichtig: die Anzahl der Zähler/Zeiger hängt nicht von der Länge der Eingabe ab.
    Die Anzahl der für die Simulation benötigten Zähler und Zeiger liegt demnach in LogSpace.


\subsubsection*{Aufgabe 3}
    Es sei $L := \{\ a^{n}\ |\ n \in \N\ \MT{ist\ keine\ Primzahl} \}$. Zeigen Sie, dass $L \in \NP$ gilt. \\

    \LOES Demnach ist $L = \{ \epsilon, a, aaaa, aaaaaa, \dots \}$. Wir nutzen den Teiler von $n$ als Zertifikat. \\
    Ein nicht-deterministisches Entscheidungsverfahren für $L$, welches in polynomieller Zeit läuft ist folgendes: $M =$ bei Eingabe $a^{n}$:
    \begin{itemize}
        \item rate $p \in \N$ mit $1 < p < n$ (es gibt $\sqrt{n}$ viele $p$)
        \item prüfe ob $p$ ein Teiler von $n$ ist
        \item falls ja, akzeptiere, ansonsten verwerfe
    \end{itemize}

    Warum $L(M) = L$? Für jedes $a^{n} \in L$ gibt es mindestens einen akzeptierenden Lauf von $M$ auf $a^{n}$ und für $a^{n} \not\in L$ verwirft sie stets.
    Ist $M$ polynomiell zeitbeschränkt? Ja, denn Test lässt sich in polynomieller Zeit ausführen. Damit ist $L \in \NP$. Sogar $L \in P$, wenn einfach alle Zahlen durchprobiert werden.
    Der Primzahltest ist in $P$, wird jedoch komplexer bei der Kodierung ($log\ n \to n^{2}$).


\subsubsection*{Aufgabe 4}
    Zeigen Sie: ist $P = \NP$, dann gibt es einen Algorithmus, der in polynomieller Zeit für jede erfüllbare aussagenlogische Formel eine erfüllende Belegung findet. \\

    \LOES Idee ist die binäre Suche mit Teilformeln. \\
    Sei $\varphi$ eine aussagenlogische Formel mit Variablen $x_{1}, \dots, x_{n}$. Angenommen $\varphi$ ist erfüllbar. Betrachte die Formel $\varphi [x_{1} \leftarrow True ]$. Ist diese Formel erfüllbar (da $P = \NP$ kann hier SAT verwendet werden), setze $\beta(x_{1}) := \MT{True}$, ansonsten setze $\beta(x_{1}) := \MT{False}$. Berechne dann rekursiv eine erfüllende Belegung $\beta'$ für $\varphi [x_{1} \leftarrow \beta(x_{1})]$. Dann ist $\beta$ erweitert um $\beta'$ eine erfüllende Belegung für $\varphi$. \\

    Was ist die Laufzeit dieses Algorithmus? Da $P = \NP$ gibt es ein Polynom $p(n)$, welches die Laufzeit für den Erfüllbarkeitstest nach oben abschätzt. Dann läuft der Algorithmus oben in Zeit $O(n \cdot p(|\varphi|)) = O(|\varphi| \cdot p(|\varphi|))$, also in polynomieller Zeit in der Größe von $\varphi$. \\

    Wichtig: Backtracking ist hier nicht notwenig, da SAT alle weiteren Belegungen nach einer Belegung prüft.
    Klappt auch für 3SAT und CLIQUE.

\input{includes/ueb/06}
\input{includes/ueb/07}
\newpage
\input{includes/ueb/08}

\newpage
\authorHead{Schmittmann}
\subsection*{Übung 9 (Prädikatenlogik)}
\subsubsection*{Aufgabe 1}
Welche der angegebenen Strukturen sind Modelle der folgenden Formel?
\begin{equation*}
\forall x.p(x,x) \land \forall x,y.((p(x,y) \land p(y,x)) \to x \approx y) \land \forall x, y, z.((p(x,y) \land p(y,z)) \to p(x,z))
\end{equation*}
\begin{enumerate}
\item $\INT_1$ mit Grundmenge $\N$ und $p^{\INT_1} = \{(m,n) \mid m < n\}$;
\item $\INT_2$ mit Grundmenge $\N$ und $p^{\INT_2} = \{(m,n + 1) \mid n \in \N \}$;
\item $\INT_3$ mit Grundmenge $\N$ und $p^{\INT_3} = \{(m,n) \mid m teilt n\}$;
\item $\INT_4$ mit Grundmenge $\SIGS$ für ein Alphabet $\Sigma$ und $p^{\INT_4} = \{(x,y) \mid x \text{ ist Präfix von } y \}$;
\item $\INT_5$ mit Grundmenge $\POT(M)$ für eine Menge $M$ und $p^{\INT_5} = \{(X,Y) \mid X \subseteq Y \}$;
\end{enumerate}
\LOES 
\begin{equation*}
\underbrace{\forall x.p(x,x)}_{\substack{{p(x,x) = \top} \\ \\ \text{p wird als reflexive} \\ \text{Relation interpretiert}}} \land \underbrace{\forall x,y.((p(x,y) \land p(y,x)) \to x \approx y)}_{\substack{{x \leq y, y \leq x, x=y} \\ \\ \text{p wird als antisymmetrische} \\ \text{Relation interpretiert}}} \land \underbrace{\forall x, y, z.((p(x,y) \land p(y,z)) \to p(x,z))}_{\substack{\text{transitivität} \\ \\ \text{p wird als transitive} \\ \text{Relation interpretiert}}}
\end{equation*}
$\Rightarrow$ Die gesamte Formel beschreibt die Theorie der Ordnungsrelation.
\begin{enumerate}[leftmargin=1cm]
\item[Zu a)] Kein Modell, denn $(2,2) \notin p^{\INT_1}$
\item[Zu b)] Kein Modell, denn $(1,2),(2,3) \in p^{\INT_2}$, aber $(1,3) \notin p^{\INT_2}$.
\begin{align*}
\text{\f{Genauer:}} \\
\text{Sei } \ZUW: &x \mapsto 1 \text{, dann gilt } \INT_2, \ZUW\#p(x,y) \land p(y,z) \to p(x,z) \\
& y \mapsto 2 \text{, also folgt } \INT_2\#\forall x, y, z.(p(x,y) \land p(y,z) \to p(x,z)) \\
& z \mapsto 3 \text{, und damit ist } \INT_2 \text{ kein Modell der Formel}  
\end{align*}
\item[Zu c)] Ist ein Modell, denn Teilbarkeit ist eine Ordnungsrelation auf $\N$.
\item[Zu d)] Ist ein Modell, denn Präfixrelation ist eine Ordnungsrelation auf $\N$.
\item[Zu e)] Ist ein Modell, denn $\subseteq$ ist eine Ordnungsrelation auf $\N$.
\end{enumerate}

\subsubsection*{Aufgabe 2}
\begin{enumerate}
\item Geben Sie eine erfüllbare Formel in Prädikatenlogik mit Gleichheit an, so dass alle Modelle 
	\begin{enumerate}[label=\roman*)]
	\item höchstens drei, \\
	\LOES $F_{\leq 3} := \exists x,y,z.\forall w.(w \approx x \lor w \approx y \lor w \approx z)$
	\item mindestens drei,\\
	\LOES $F_{\geq 3} := \exists x,y,z.(\underbrace{x \not\approx y}_{= \neg (x \approx y)} \land y \not\approx z \land x \not\approx z)$
	\item genau drei \\
	\LOES $F_{=3} := F_{\leq 3} \land F_{\geq 3}$
	\end{enumerate}
	Elemente in der Grundmenge besitzen.
\item Geben Sie je eine erfüllbare Formel in Prädikatenlogik mit Gleichheit an, so dass das zweistellige Relationensymbol $p$ in jedem Modell als der Graph einer

	\begin{enumerate}[label=\roman*)]
	\item injektiven Funktion, \\
	\LOES $F_{fun} := \underbrace{\forall x.\exists y. p(x,y)}_{\substack{{p \text{ wird als linkstotale}} \\ \text{Relation interpretiert}}} \land \underbrace{\forall x, y, z.(p(x,y) \land p(x,z) \to y \approx z)}_{\substack{{p \text{ wird als rechtseindeutige}} \\ \text{Relation interpretiert}}}$ \\
	$F_{inj} := F_{fun} \land \forall x,y,z.(p(x,z) \land p(y,z) \to x \approx y)$
	\item surjektiven Funktion, \\
	\LOES $F_{sur} := F_{fun} \land \forall y. \exists x.p(x,y)$
	\item bijektiven Funktion \\	
	\LOES $F_{bij} := F_{inj} \land F_{sur}$
	\end{enumerate}
	interpretiert wird. \\
	(Der Graph einer Funktion $f: A \to B$ ist die Relation $\{(x,y) \in A \times B \mid f(x) = y \}$.)
\end{enumerate}

\subsubsection*{Aufgabe 3}
Welche der folgenden Aussagen sind wahr? Begründen Sie Ihre Antwort.
\begin{enumerate}
\item Sind $\Gamma$ und $\Gamma'$ Mengen von prädikatenlogischen Formeln, dann folgt aus $\Gamma \subseteq \Gamma'$ und $\Gamma \models F$ auch $\Gamma' \models F$. \\
\LOES \textcolor{green}{Ja}, $\Gamma \subseteq \Gamma'$ und $\overbrace{\Gamma \models F}^{\forall \text{ Strukturen } \INT: \INT \models \Gamma \Rightarrow \INT \models F}$  impliziert $\overbrace{\Gamma' \models F}^{\forall \text{ Strukturen } \INT: \INT \models \Gamma' \Rightarrow \INT \models F}$. \\
Die Aussage gilt: Sei $\INT \models \Gamma'$. Wegen $\Gamma \subseteq \Gamma'$ folgt $\INT \models \Gamma$. \\
Mit $\Gamma \models F$ folgt $\INT \models F$. Also insgesamt haben wir $\Gamma' \models F$ gezeigt.
\item Jede aussagenlogische Formel ist eine prädikatenlogische Formel. \\
\LOES \textcolor{green}{Ja}, mit der in der VL gezeigten Einbettung von Aussagenlogik in Prädikatenlogik.
\item Eine prädikatenlogische Formel $F$ ist genau dann allgemeingültig, wenn $\neg F$ unerfüllbar ist. \\
\LOES \textcolor{green}{Ja}, denn: 
\begin{align*}
F \text{ allgemeingültig } &\Leftrightarrow \forall \text{ Strukturen } \INT:\INT \models F \\
& \Leftrightarrow \forall \text{ Strukturen } \INT: \INT \not\models \neg F \\
& \Leftrightarrow \neg F \text{ unerfüllbar }
\end{align*}
\item Es gilt 
\begin{equation*}
\{\forall x,y.(p(x,y) \to p(y,x)), \forall x,y,z.((p(x,y) \land p(y,z)) \to p(x,z))\} \models \forall x.p(x,x).
\end{equation*}
\LOES \textcolor{orange}{Falsch!} Übersetzt wird hier gefragt, ob aus Symmetrie und Transitivität einer binären Relation stets ihre Reflexivität folgt. \\
Gegenbeispiel: $\INT := (\{d\},\{p \mapsto \emptyset \})$
\end{enumerate}
\subsubsection*{Aufgabe 4}
\label{U9-4}
Formalisieren Sie Bertrand Russells Barbier-Paradoxon 
\begin{center}
\textit{Der Barbier rasiert genau diejenigen Personen, die sich nicht selbst rasieren.}
\end{center}
als eine prädikatenlogische Formel und zeigen Sie, dass diese unerfüllbar ist. \\\\
\LOES Wir verwenden die Menge $C := \{Barbier\}$ als Menge der Konstanten und die Menge $\f{P} := \{rasiert\}$ als Menge der Prädikatensymbole. Nun formulieren wir die gegebene Aussage in Prädikatenlogik wie folgt:
\begin{equation*}
F := \forall x.(\neg rasiert(x,x) \leftrightarrow rasiert(barbier,x))
\end{equation*}
Wir zeigen: $F$ ist unerfüllbar. \\\\
Sei $\INT$ eine Interpretation. Dann ist zu zeigen, dass $\INT \not\models F$. 
\begin{align*}
\INT \models F &\Leftrightarrow \INT \models \forall x.(rasiert(barbier,x) \leftrightarrow \neg rasiert(x,x)) \\
&\Leftrightarrow \text{ Für alle } \delta_x \in \Delta^{\INT} \text{ gilt, dass } \INT, \{ x \mapsto \delta_x \} \models rasiert(barbier, x)  \leftrightarrow \neg rasiert(x,x)
\end{align*}
Für das Element $\delta_x$ mit $barbier^{\INT} = \delta_x$ gilt:
\begin{align*}
\INT, \{x \mapsto \delta_x\} \models rasiert(barbier,x) \leftrightarrow \neg rasiert(x,x) 
\Leftrightarrow \underbrace{(barbier^{\INT}, \delta_x}_{=\delta_x} \in rasiert^{\INT} \Leftrightarrow (\delta_x, \delta_x) \not\in rasiert^{\INT}
\end{align*}
Es ergibt sich der Widerspruch $(\delta_x, \delta_x) \in rasiert^{\INT} \Leftrightarrow (\delta_x, \delta_x) \not\in rasiert^{\INT}$ und damit ist $\INT$ kein Modell von $F$. Weil $\INT$ beliebig, folgt die Unerfüllbarkeit von $F$.

\newcommand{\HRule}[2]{\par
  \vspace*{\dimexpr-\parskip-\baselineskip+#2}
  \noindent\rule{#1}{0.2mm}\par
  \vspace*{\dimexpr-\parskip-.5\baselineskip+#2}}

\subsection*{Übung 10 (Skolemform)}
\subsubsection*{Aufgabe 1}
Bestimmen Sie zu jeder der folgenden Formeln eine äquivalente bereinigte Formel in Pränexform.
\begin{enumerate}
\item $\forall x.(p(x,x) \leftrightarrow \neg \exists y.q(x,y))$ \\
\LOES 
\begin{align*}
& \forall x.(p(x,x) \leftrightarrow \neg \exists y.q(x,y)) \\ 
\equiv\, & \forall x.(p(x,x) \to \neg \exists y.q(x,y)) \land (\neg \exists y.q(x,y) \to p(x,x))) \\
\equiv\, & \forall x.((\neg p(x,x) \lor \underbrace{\neg \exists y}_{\forall y.\neg}.q(x,y)) \land (\exists y'.q(x,y') \lor p(x,x))) \\
\equiv\, & \forall x,y.\exists y'.((\neg p(x,x) \lor \neg q(x,y)) \land (q(x,y') \lor p(x,x)))
\end{align*}
\item $\forall x.p(f(x,x)) \lor (q(x,z) \to \exists x.p(g(x,y,z)))$ \\
\LOES 
\begin{align*}
& \forall x.p(f(x,x)) \lor (q(x,z) \to \exists x.p(g(x,y,z))) \\
\equiv\, & \forall x'.(p(f(x',x')) \lor (\neg q(x,z) \lor \exists x''.p(g(x'',y,z)))) \\
\equiv\, & \forall x'.\exists x''.(p(f(x',x')) \lor \neg q(x,z) \lor p(g(x'',y,z)))
\end{align*}
\item $\forall x.p(x) \land (\forall y.\exists x.q(x,g(y)) \to \exists y.(r(f(y)) \lor \neg q(y,x)))$ \\
\LOES
\begin{align*}
& \forall x.p(x) \land (\forall y.\exists x.q(x,g(y)) \to \exists y.(r(f(y)) \lor \neg q(y,x))) \\
\equiv\, & \forall x''.p(x'') \land (\underbrace{\neg\forall y.\exists x'.q(x',g(y))}_{\equiv \exists y.\forall x'.\neg q(x',g(y))} \lor \, \exists y'.(r(f(y')) \lor \neg q(y',x)) \\
\equiv\, & \forall x''.\exists y.\forall x'.\exists y'.(p(x'') \land (q(x',g(y)) \lor r(f(y') \lor \neg q(y',x)))
\end{align*}
(Tipp für Skolemform: $\exists$-Quantoren möglichst nach links ziehen.)
\end{enumerate}
\subsubsection*{Aufgabe 2}
Bestimmen Sie zu jeder der folgenden Formeln eine erfüllbarkeitsäquivalente bereinigte Formel in Skolemform.
\begin{enumerate}
\item $p(x) \lor \exists x.q(x,x) \lor \forall x.p(f(x))$ \\
\LOES \begin{align*}
& p(x) \lor \exists x.q(x,x) \lor \forall x.p(f(x)) \\
\equiv\, & \exists u.\forall v.(p(x) \lor q(u,u) \lor p(f(v))) \\
\rightarrow_{skolemform}\, & \forall v.(p(x) \lor q(c,c) \lor p(f(v))) \text{,}
\end{align*}
wobei $c$ eine neue Konstante ist.
\item $\forall x. \exists y.q(f(x),g(y)) \land \forall x.(p(x,y,y) \lor q(h(y),x))$ \\
\LOES 
\begin{align*}
& \forall x. \exists y.q(f(x),g(y)) \land \forall x.(p(x,y,y) \lor q(h(y),x)) \\
\equiv\, & \forall x.\exists u.\forall v.(q(f(x),g(u)) \land (p(v,y,y) \lor q(h(y),v))) \\
\rightarrow_{skolemform}\, & \forall x,v.(q(f(x), g(l(x)) \land (p(v,y,y) \lor q(h(y),v))) \text{,}
\end{align*}
wobei $l$ ein neues ein-stelliges Funktionssymbol ist.
\item $\forall x. \forall x.(p(x) \leftrightarrow q(x,x)) \lor \exists x.\forall y.(q(x,g(y,z)) \land \exists z.q(z,z))$ \\
\LOES 
\begin{align*}
& \forall x. \forall x.(p(x) \leftrightarrow q(x,x)) \lor \exists x.\forall y.(q(x,g(y,z)) \land \exists z.q(z,z)) \\
\equiv\, & \forall x.\exists u.\forall v.\exists w(((p(x) \land q(x,x)) \lor (\neg p(x) \land \neg q(x,x) \lor (q(n,g(v,z)) \land q(w,w))) \\
\rightarrow_{skolemform}\, & \forall x,v.(((p(x) \land q(x,x)) \lor (\neg p(x) \land \neg q(x,x))) \lor (q(f(x), g(v,z)) \land q(h(x,v), h(x,v)))) \text{,}
\end{align*}
wobei $f$ und $h$ neue Funktionssymbole sind.
\end{enumerate}

\subsubsection*{Aufgabe 3}
Gegeben sind die folgenden Formeln in Skolemform. 
\begin{align*}
F &= \forall x,y,z.p(x,f(y),g(z,x)), \\
G &= \forall x,y.(p(a,f(a,x,y)) \lor q(b)),
\end{align*}
wobei $a$ und $b$ Konstanten sind.
\begin{enumerate}
\item Geben Sie die zugehörigen Herbrand-Universen $\Delta_F$ und $\Delta_G$ an. \\
\LOES 
\begin{align*}
\Delta_F &= \{a, f(a), g(a,a), f(f(a)), g(f(a),f(a)), \dots \} \\
\Delta_G &= \{a, b, f(a,a,a), f(a,a,b), f(a,b,a), f(b,a,a), \dots, f(f(a,a,a),f(b,a,b),f(b,a,a)), \dots \} \\
\text{Wir können } &\Delta_F \text{ und } \Delta_G \text{ auch rekursiv wie folgt charakterisieren:} \\
\Delta_F &= \{a\} \cup \{f(t),g(t,u) \mid t,u \in \Delta_F \} \\
\Delta_G &= \{a,b\} \cup \{f(s,t,u) \mid s,t,u \in \Delta_G\}
\end{align*}
\item Geben Sie je ein Herbrand-Modell an oder begründen Sie, warum kein solches existiert. \\
\LOES Für $F: a^{\INT} := a,\quad f^{\INT}(t) := f(t),\quad g^{\INT}(s,t) := g(s,t) \text{ mit } s,t \in \Delta_F$ \\
Definiere noch: $p^{\INT} := \Delta_F^3$, alternativ: $p^{\INT} := \{(r,f(s),g(t,r)) \mid r,s,t \in \Delta_F \}$. \\
Dann ist die Herbrand-Interpretation $(\Delta_F, \cdot^{\INT})$ ein Modell von $F$.
\HRule{3cm}{3mm}
Für $G: a^{\INT} := a,\quad b^{\INT} := b,\quad f^{\INT}(v,s,t) := f(v,s,t)$. \\\
Definiere noch: $p^{\INT} := \{(a,f(a,s,t)) \mid s,t \in \Delta_G \}$, $q^{\INT} := \{b\}$. \\
Dann ist $(\Delta_G, \cdot^{\INT})$ ein Herbrand Modell von $G$.
\item Geben Sie die Herbrand-Expansion $\HE(F)$ und $\HE(G)$ an. \\
\LOES 
\begin{align*}
\HE(F) &= \{p(a,f(a),g(a,a)), \dots \} \\
&= \{p(r,f(s),g(t,r)) \mid r,s,t \in \Delta_F \} \\
\HE(G) &= \{ p(a,f(a,f(a,a,a),b)) \lor q(b), \dots \} \\
&= \{ p(a,f(a,s,t)) \lor q(b) \mid s,t \in \Delta_G \}
\end{align*}
\end{enumerate}

\subsubsection*{Aufgabe 4}
Zeigen Sie, dass Allgemeingültigkeit von Formeln der Prädikatenlogik erster Stufe in Skolemform entscheidbar ist. \\
\LOES Es sei $F$ eine quantorenfreie Formel mit Variablen $x_1, \dots, x_n$. Dann gilt
\begin{align*}
\forall x_1, \dots, x_n. F ist allgemeingültig \Leftrightarrow & \exists x_1, \dots, x_n.\neg F \text{ ist unerfüllbar} \\
\Leftrightarrow & \neg F[x_1/a_1, \dots, x_n/a_n] \text{ ist unerfüllbar} \\
& \text{(Skolemisierung mit Konstanten } a_1, \dots, a_n \text{).}
\end{align*}
Es ist also $\forall x_1, \dots, x_n.F$ allgemeingültig genau dann, wenn $\neg F[x_1/a_1, \dots, x_n/a_n]$ unerfüllbar ist. \\
Letzteres ist aber essentiell eine aussagenlogische Formel, und deren Erfüllbarkeit ist entscheidbar.
\newpage
\subsection*{Übung 11 (Allgemeinster Unifikator, Resolution)}
\subsubsection*{Aufgabe 1}
Bestimmen Sie jeweils einen allgemeinsten Unifikator der folgenden Gleichungsmengen, oder begründen Sie, warum kein allgemeinster Unifikator existiert. Verwenden Sie hierfür den Algorithmus aus der Vorlesung. Dabei sind $x,y$ Variablen und $a,b$ Konstanten.
\begin{enumerate}
\item $\{ f(x) \dot{=} g(x,y), y \dot{=} f(a) \}$ \\
\LOES Der Algorithmus besteht aus 4 Regeln: 
\begin{itemize}
\item Löschen: $t=t$
\item Orientieren: $t=x \mapsto x=t$
\item Zerlegen: $f(t_1,\dots,t_n) = f(s_1,\dots,s_n) \mapsto t_1=s_1,\dots, t_n=s_n$
\item Einsetzen (Eliminieren): $x=t$ in anderer Gleichung einsetzen, falls $x$ nicht vorkommt.
\end{itemize}
\begin{equation*}
\{ f(x) = g(x,y), y = f(a) \} \rightarrow_{einsetzen} \{ f(x) = g(x,f(a)), y = f(a) \}
\end{equation*}
Keine weitere Regel anwendbar und Menge nicht in gelöster Form. \\
Also gibt es keinen (allgemeinsten) Unifikator.
\item $\{ f(g(x,y)) \dot{=} f(g(a,h(b))) \}$ \\
\LOES 
\begin{equation*}
\{ f'(g(x,y)) = f'(g(a,h(b))) \} \rightarrow_{zerlegen} \{g(x,y) = g(a,h(b)) \rightarrow_{zerlegen} \{x=a, y=h(b)\}
\end{equation*}
Keine weitere Regel anwendbar und Menge in gelöster Form. \\
Ein allgemeinster Unifikator ist $\{x \mapsto a, y \mapsto h(b) \}$
\item $\{ f(x,y) \dot{=} x, y \dot{=} g(x) \}$ \\
\LOES
\begin{equation*}
\{ f(x,y) = x, y = g(x) \} \rightarrow_{orientieren} \{x=f(x,y), y=g(x)\} \rightarrow_{einsetzen} \{x = f(x,g(x)), y= g(x) \}
\end{equation*}
Keine weitere Regel anwendbar. \\
Menge nicht in gelöster Form.
\item $\{ f(g(x),y) \dot{=} f(g(x),a), g(x) \dot{=} g(h(a)) \}$ \\
\LOES
\begin{equation*}
\{ f(g(x),y) = f(g(x),a), g(x) = g(h(a)) \} \rightarrow_{zerlegen} \{ g(x) = g(x), y=a, g(x) = g(h(a)) \} \rightarrow_{zerlegen} \{ g(x) = g(x), y=a, x=h(a) \} \rightarrow_{löschen} \{y=a, x=h(a) \}
\end{equation*}
Fertig, in gelöster Form, ein allg. Unifikator ist $\{y \mapsto a, x \mapsto h(a) \}$.
\item Zusatz. $\{x\dot{=}a, x\dot{=}h(a)\}$  \\
\LOES
\begin{equation*}
\{x=a, x=h(a)\} \rightarrow_{einsetzen} \{h(a) = a, x=h(a) \}
\end{equation*}
Nicht in gelöster Form.
\item Zusatz. $\{x\dot{=}z, y\dot{=}h(z)\}$  \\
\LOES
In gelöster Form. Unifikator $\{ x \mapsto z, y \mapsto h(z) \}$
\end{enumerate}
\subsubsection*{Aufgabe 2}
Zeigen Sie mittels prädikatenlogischer Resolution folgende Aussagen: 
\begin{enumerate}
\item Die Aussage \glqq Der Professor ist glücklich, wenn alle seine Studenten Logik mögen\grqq \\
hat als Folgerung \glqq Der Professor ist glücklich, wenn er keine Studenten hat\grqq. \\
\LOES Wir verwenden folgendes Vokabular: $\{\text{glücklich}/1, \text{magLogik}/1, \text{student}/1, \text{prof}(Konstante)\}$ 
\begin{align*}
F_1 &= \forall x.(\text{student}(x) \to \text{magLogik}(x)) \to \text{glücklich}(prof) \\
F_2 &= \neg \exists x.\text{student}(x) \to \text{glücklich}(prof)
\end{align*} 
\underline{Ziel}: Zeige $F_1 \models F_2$. Zeige dazu, $\{F_1, \neg F_2\}$ ist unerfüllbar. \\
\underline{Normalformen}: 
\begin{align*}
F_1 &= \forall x.(\neg\text{student}(x) \lor \text{magLogik}(x) \to \text{glücklich}(prof) \\
&\equiv \neg \forall x.(\neg \text{student}(x) \lor \text{magLogik}(x) \lor \text{glücklich}(prof) \\
&\equiv \exists x.((\text{student}(x) \land \neg \text{magLogik}(x)) \lor \text{glücklich}(prof) \qquad \text{Pränexform} \\
&=_{Skolem} (\text{student}(c) \land \neg \text{magLogik}(c)) \lor \text{glücklich(prof)} \\ 
&\equiv (\text{student}(c) \lor \text{glücklich}(prof)) \land (\neg \text{magLogik}(c) \lor \text{glücklich}(prof)) \\
\neg F_2 &= \neg (\neg \exists x.\text{student}(x) \to \text{glücklich}(prof)) \\
&\equiv \neg (\exists x.\text{student}(x) \lor \text{glücklich}(prof)) \\
&\equiv \forall x.(\neg \text{student}(x) \land \neg \text{glücklich}(prof))
\end{align*}
Klauselmenge: 
\begin{align*}
\{&\{\text{student}(c), \text{glücklich}(prof)\}^{(1)}, \{\neg\text{magLogik}(c),\text{glücklich}(prof)\}^{(2)}, \\ 
&\{\neg\text{student}(x)\}^{(3)}, \{\neg\text{glücklich}(prof)\}^{(4)}\}
\end{align*}
Resolution:
\begin{align*}
(5) &= (1) + (4) \text{ ergibt } \{\text{student}(c)\} \\
(6) &= (3) + (5) \text{ mit Unifikator } \{x \mapsto c\} \text{ ergibt } \bot
\end{align*}
Also ist $\{F_1, \neg F_2\}$ unerfüllbar und es gilt $F_1 \models F_2$
\item Die Formulierung des Barbier-Paradoxons aus Aufgabe 4 von Blatt 9 [\ref{U9-4}] ist unerfüllbar. \\
\LOES 
\begin{align*}
F &= \forall x.(rasiert(barbier, x) \leftrightarrow \neg rasiert(x,x) \\
&= \forall x.((rasiert(barbier,x) \lor rasiert(x,x)) \land (\neg rasiert(barbier,x) \lor \neg rasiert(x,x)))
\end{align*}
Klauseln:
\begin{equation*}
\{\{rasiert(barbier,x),rasiert(x,x)\}^{(1)}, \{\neg rasiert(barbier, x), \neg rasiert(x,x)\}^{(2)}\}
\end{equation*}
Resolution: $(1) + (2)$ mit Unifikator $\{x \mapsto barbier\}$ ergibt $\bot$
\item In Aufgabe V [\ref{REP3-V}]folgt die letzte Aussage aus den ersten drei. (Zur Vereinfachung darf hier angenommen werden, dass alle Individuen Drachen sind.) \\
\LOES
\begin{align*}
F_1 &:= \forall x.(\forall y.(\text{kind}(x,y) \to \text{fliegen}(y)) \to \text{glücklich}(x)) \\
F_2 &:= \forall x.(\text{grün}(x) \to \text{fliegen}(x)) \\
F_3 &:= \forall x.(\exists y(\text{kind}(y, x) \land \text{grün}(y)) \to \text{grün}(x)) \\
F_4 &:= \forall x.(\text{grün}(x) \to \text{glücklich}(x))
\end{align*}
Klauselmenge (nach umformulieren):
\begin{align*}
\{&\{\text{kind}(x,f(x)), \text{glücklich}(x)\}^{(1)},\{\neg\text{fliegen}(f(x)), \text{glücklich}(x)\}^{(2)}, \\
&\{\neg\text{grün}(x), \text{fliegen}(x)\}^{(3)}, \{\neg\text{kind}(y,x), \neg\text{grün}(y), \text{grün}(x) \}^{(4)}, \\
&\{\text{grün}(c)\}^{(5)}, \{\neg\text{glücklich}(c)\}^{(6)}\}
\end{align*}
Resolution: 
\begin{align*}
(7) = (1)+(4) & \text{ Variante von } (4): \{\neg \text{kind}(z,w), \neg\text{grün}(z),\text{grün}(w)\} \\
& \text{ Unifikator } \{z \mapsto x, w \mapsto f(x)\} \text{ ergibt Resolvente } \{\text{glücklich}(x),\neg\text{grün}(x),\text{grün}(f(x))\} \\
(8) = (5)+(7) &,\{x \mapsto c\} \text{ ergibt } \{\text{glücklich}(c),\text{grün}(f(c))\} \\
(9) = (8)+(6) & \text{ ergibt } \{\text{grün}(f(c))\} \\
(10) = (3)+(9) & \text{ mit } \{x \mapsto f(c)\} \text{ ergibt } \{\text{fliegen}(f(c))\} \\
(11) = (10)+(2) & \text{ mit } \{x \mapsto c\} \text{ ergibt } \{\text{glücklich}(c)\} \\
(12) = (11)+(6) & \text{ ergibt } \bot
\end{align*}
Also gilt $\{F_1, F_2, F_3\} \models F_4$
\end{enumerate}


    \newpage
    \authorHead{Pataky}
    \include{includes/rep1}
    \include{includes/rep2}

    \authorHead{Schmittmann}
    \subsection*{Repetitorium III}
\subsubsection*{Aufgabe I}
Geben Sie für die Formel
\begin{equation*}
F = \forall x.\exists y.(p(c_1,z) \land (q(x,c_2,z)) \lor p(c_2,y))),
\end{equation*}
wobei $c_1,c_2$ Konstanten sind, folgendes an:
\begin{enumerate}
        \item die Menge der Teilformeln; \\
        \LOES
        \begin{itemize}
        \item $F$
        \item $\exists y.(p(c_1,z) \land (q(x, c_2, z) \lor p(c_2,y)))$
        \item $p(c_1,z) \land (q(x,c_2,z) \lor p(c_2, y))$
        \item $p(c_1, z)$
        \item $q(x,c_2,z)$
        \item \dots
        \end{itemize}
        \item die Menge aller Terme; \\
        \LOES $\{ x, y, z, c_1, c_2 \}$
        \item die Menge aller Variablen, mit Unterscheidung freier und gebundener Variablen; \\
        \LOES $\{ x (gebunden), y (gebunden), z (frei) \}$
        \item eine Interpretation $\INT$ und eine Zuweisung $\ZUW$ für $\INT$, so dass $\INT,\ZUW \models F$. \\
\LOES Für die folgende Interpretation $\INT$ und zugehörige Zuweisung $\ZUW$ gilt $\INT, \ZUW \models F$:
\begin{itemize}
        \item $\Delta^{\INT} := \{\delta \}$
        \item $c_1^{\INT} := \delta$
        \item $c_2^{\INT} := \delta$
        \item $p^{\INT} := \{(\delta,\delta)\}$
        \item $q^{\INT} := \{(\delta,\delta,\delta)\}$
        \item $\ZUW := \{ z \mapsto \delta \}$
\end{itemize}
\f{Achtung:} Es ist keine Interpretation der gebundenen Variablen $x$ und $y$ notwendig. \\

Es gilt:
\begin{align*}
&\INT, \ZUW \models \forall x.\exists y.(p(c_1,z) \land (q(x,c_2,z)) \lor p(c_2,y))) \\
\text{gdw.} & \text{ für alle } \delta_x \in \Delta^{\INT} \text{ existiert ein } \delta_y \in \Delta^{\INT} \text{, sodass gilt} \\
&\INT,\ZUW [x \mapsto \delta_x, y \mapsto \delta_y ] \models p(c_1, z) \text{, und} \\
&\INT,\ZUW [x \mapsto \delta_x, y \mapsto \delta_y ] \models q(x,c_2,z) \text{ oder } \INT,\ZUW [x \mapsto \delta_x, y \mapsto \delta_y ] \models p(c_2,y)
\end{align*}

Die umgeformte letzte Bedingung ist tatsächlich erfüllt, denn für $\delta_x$ gibt es nur die Möglichkeit $\delta_x=\delta$, und wir sehen auch schnell, dass wir stets $\delta_y=\delta$ wählen können. \\
Es gilt dann nämlich für die Zuordnung
\begin{equation*}
\ZUW^* := \ZUW[x \mapsto \delta, y \mapsto \delta] = \{x \mapsto \delta, y \mapsto \delta, z \mapsto \delta \}
\end{equation*}
folgendes:
\begin{align*}
& \INT, \ZUW^* \models p(c_1, z) \text{, und } \INT, \ZUW^* \models q(x,c_2,z) \text{ oder } \INT,\ZUW^* \models p(c_2,y) \\
\text{gdw. } & (c_1^{\INT}, \ZUW^*(z)) \in p^{\INT} \text{, und } (\ZUW^*(x), c_2^{\INT}, \ZUW^*(z)) \in q^{\INT} \text{ oder } (c_2^{\INT}, \ZUW^*(y)) \in p^{\INT} \\
\text{gdw. } & (\delta, \delta) \in \{(\delta, \delta)\} \text{, und } (\delta, \delta, \delta) \in \{(\delta, \delta, \delta)\} \text{ oder } (\delta,\delta) \in \{(\delta, \delta) \}
\end{align*}
\end{enumerate}


\subsubsection*{Aufgabe II}
Zeigen Sie die folgenden Aussagen:
\begin{enumerate}
        \item Es gilt $\{F\} \models G$ genau dann, wenn $F \to G$ allgemeingültig ist.\\
        \LOES Es gilt
        \begin{align*}
        &\{F\} \models G \\
        \text{gdw. } &\text{jedes Modell von } F \text{ ist ein Modell von } G \\
        \text{gdw. } &\text{jede Interpretation ist kein Modell von } F \text{ oder ein Modell von } G \\
        \text{gdw. } &\text{jede Interpretation ist ein Modell von } \lnot F \text{ oder ein Modell von } G \\
        \text{gdw. } &\text{jede Interpretation ist ein Modell von } \lnot F \lor G  \\
        \text{gdw. } &\text{jede Interpretation ist ein Modell von } F \to G \\
        \text{gdw. } & F \to G \text{ ist allgemeingültig. }
        \end{align*}
        \item Es gilt $\{F_1,\dots,F_k\} \models G$ genau dann, wenn $\land_{i=1}^k F_i \to G$ allgemeingültig ist. \\
        \LOES Es gilt die Äquivalenz
        \begin{equation*}
        \{F_1, \dots ,F_k\} \models G \text{ gdw. } \{\land_{i=1}^k F_i\} \models G,
        \end{equation*}
        denn die Formelmengen $\{F_1,\dots ,F_n \}$ und $\{ \land_{i=1}^n F_i \}$ haben die gleichen Modelle. \\
        Also folgt mit dem ersten Teil sofort, dass
        \begin{equation*}
        \{ F_1, \dots , F_k \} \models G \text{ gdw. } \land_{i=1}^k F_i \to G \text{ allgemeingültig ist.}
        \end{equation*}
\end{enumerate}

\subsubsection*{Aufgabe III}
Seien $F$, $G$ Formeln und $x$ eine Variable. Zeigen Sie, dass dann gilt
\begin{equation*}
\exists x.(F \to G) \equiv \forall x. F \to \exists x. G.
\end{equation*}
\LOES Es gilt nach Definition von $\to$ und Folie 21 von \vl{TIL 16}
\begin{align*}
& \exists x.(F \to G) \\
\equiv \,& \exists x.(\lnot F \lor G) \\
\equiv \,& \exists x.\lnot \lor \exists x.G \\
\equiv \,& \lnot \forall x.F \lor \exists x.G \\
\equiv \,& \forall x.F \to \exists x.G.
\end{align*}

\subsubsection*{Aufgabe IV}
Welche der folgenden Aussagen sind wahr? Begründen Sie Ihre Antwort.
\begin{enumerate}
\item Jede Formel in Pränexform ist in Skolemform.\\
\LOES \textcolor{orange}{Die Aussage ist \f{falsch}}. \\
Die Formel $\exists x.p(x)$ ist in Pränexform, aber nicht in Skolemform.
\item Jede Formel in Skolemform ist in Pränexform. \\
\LOES \textcolor{green}{Die Aussage ist \f{wahr}}.\\
\f{Nach Definition ist jede Skolemform} von der Form $\forall x_1. \dots \forall x_k. F$ für eine quantorenfreie Formel $F$, d.h. \f{in Pränexform}
\item Jede Formel ist äquivalent zu einer bereinigten Formel. \\
\LOES \textcolor{green}{Die Aussage ist \f{wahr}}. \\
\f{Umbenennung von gebundenen Variablen ändert nichts} an der Interpretation der Formel.
\item Jede Formel ist äquivalent zu einer bereinigten Formel in Pränexform. \\
\LOES \textcolor{green}{Die Aussage ist \f{wahr}}, siehe Folien 7-9 von \vl{TIL, 17}.
\item Jede Formel ist äquivalent zu einer bereinigten Formel in Skolemform. \\
\LOES \textcolor{orange}{Die Aussage ist \f{falsch}}. \\
Jede Skolemform ist von der Form $F := \forall x_1, \dots, x_n.G$ wobei $x_1, \dots, x_n$ Variablen sind und $G$ eine quantorenfreie Formel ist. \\
Geschlossene Formeln dieser Art sind monoton in folgendem Sinne: Wenn $\INT$ ein Modell von $F$ ist, dann gilt auch für jede induzierte Teilinterpretation $\SUBINT$ von $\INT$, dass $\SUBINT$ ein Modell von $F$ ist. \\
Das gilt aber zum Beispiel nicht für die Formel $\exists x.p(x)$, und damit kann sie nicht äquivalent zu einer (bereinigten) Formel in Skolemform sein.
\end{enumerate}

\subsubsection*{Aufgabe V}
\label{REP3-V}
Formalisieren Sie die folgenden Aussagen in Prädikatenlogik:
\begin{enumerate}
\item Jeder Drache ist glücklich, wenn alle seine Drachen-Kinder fliegen können.
\item Grüne Drachen können fliegen.
\item Ein Drache ist grün, wenn er Kind mindestens eines grünen Drachens ist.
\item Alle grünen Drachen sind glücklich.
\end{enumerate}
Zeigen Sie, dass die letzte Aussage aus den ersten drei folgt. \\\\
\LOES Zur Formalisierung der vier Aussagen verwenden wir die folgende Menge $\f{P}$ von Prädikatensymbolen: $\f{P} := \{\text{kind}/2, \text{fliegen}/1, \text{glücklich}/1, \text{grün}/1 \}$.
\begin{enumerate}
\item Jeder Drache ist glücklich, wenn alle seine Drachen-Kinder fliegen können.\\
$F_1 := \forall x.(\forall y.(\text{kind}(x,y) \to \text{fliegen}(y)) \to \text{glücklich}(x))$
\item Grüne Drachen können fliegen. \\
$F_2 := \forall x.(\text{grün}(x) \to \text{fliegen}(x))$
\item Ein Drache ist grün, wenn er Kind mindestens eines grünen Drachens ist. \\
$F_3 := \forall x.(\exists y(\text{kind}(y, x) \land \text{grün}(y)) \to \text{grün}(x))$ \\
$\phantom{F_3}\hspace{0.3em} \equiv \forall x,y.((\text{kind}(x, y) \land \text{grün}(x)) \to \text{grün}(y))$ \\
$\phantom{F_3}\hspace{0.3em} \equiv \forall x,y.(\text{kind}(x,y) \to (\text{grün}(x) \to \text{grün}(y)))$
\item Alle grünen Drachen sind glücklich. \\
$F_4 := \forall x.(\text{grün}(x) \to \text{glücklich}(x))$
\end{enumerate}
Wir zeigen nun, dass $F_4$ eine semantische Konsequenz der drei $F_i$ ist, d.h. dass $\{F_1, F_2, F_3 \} \models F_4$ gilt.
\begin{enumerate}
\item[1)]
\begin{tabular}[t]{p{0.5\textwidth} p{0.5\textwidth}}
Sei $\delta$ ein grüner Drache, und es gelten die Aussagen (1), (2) und (3). & Sei $\INT$ eine Interpretation mit \newline
$\INT \models \{F_1, F_2, F_3 \}$, und sei $\delta \in \Delta^{\INT}$ mit $\delta \in \text{grün}^{\INT}$.
\end{tabular}
\item[2)]
\begin{tabular}[t]{p{0.5\textwidth} p{0.5\textwidth}}
Nach (3) folgt, dass jedes Kind von $\delta$ grün ist. & Mit $\INT \models F_3$ folgt \newline
$\forall \epsilon \in \Delta^{\INT}: (\delta, \epsilon) \in \text{kind}^{\INT} \Rightarrow \epsilon \in \text{grün}^{\INT}$.
\end{tabular}
\item[3)]
\begin{tabular}[t]{p{0.5\textwidth} p{0.5\textwidth}}
Mit (2) erhalten wir, dass alle Kinder von $\delta$ fliegen können. & Mit $\INT \models F_2$ folgt \newline
$\forall \epsilon \in \Delta^{\INT}: (\delta, \epsilon) \in \text{kind}^{\INT} \Rightarrow \epsilon \in \text{fliegen}^{\INT}$.
\end{tabular}
\item[4)]
\begin{tabular}[t]{p{0.5\textwidth} p{0.5\textwidth}}
Wegen (1) ist $\delta$ glücklich. & Mit $\INT \models F_1$ folgt $\delta \in \text{glücklich}^{\INT}$.
\end{tabular}
\item[5)]
\begin{tabular}[t]{p{0.5\textwidth} p{0.5\textwidth}}
Also folgt (4) aus den Aussagen (1), (2) und (3). & Weil $\INT$ und $\delta \in \Delta^{\INT}$ beliebig waren, folgt \newline
$\{F_1, F_2, F_3\} \models F_4$.
\end{tabular}
\end{enumerate}

\subsubsection*{Aufgabe VI}
Welche der folgenden Aussagen sind wahr? Begründen Sie Ihre Antwort.
\begin{enumerate}
\item Zwei prädikatenlogische Formeln $F$ und $G$ sind äquivalent, wenn die Formel $F \leftrightarrow G$ allgemeingültig ist. \\
\LOES \textcolor{green}{Die Aussage ist \f{wahr}} \\
\begin{align*}
& F \equiv G \\
\text{gdw. }& \{F\} \models G \text{ und } \{G\} \models F \\
\text{gdw. }& \emptyset \models F \to G \text{ und } \emptyset \models G \to F \\
\text{gdw. }& \emptyset \models (F \to G) \land (G \to F) \\
\text{gdw. }& \emptyset \models (F \leftrightarrow G)
\end{align*}
\item Jede erfüllbare Formel der Prädikatenlogik erster Stufe hat ein endliches Modell. \\
\LOES \textcolor{orange}{Die Aussage ist \f{falsch.}} \\
Für Abbildungen $f: A \to A$ auf einer endlichen Menge $A$ gilt, dass $f$ injektiv ist genau dann, wenn $f$ surjektiv ist. Folglich gilt für injektive und nicht surjektive Abbildungen $f: A \to A$, dass $A$ nicht endlich sein kann. \\
\f{Die Formel} $F_{\infty} := F_{inj} \land \lnot F_{sur}$ aus Aufgabe 4 vom Übungsblatt 9 ist erfüllbar, aber \f{hat kein endliches Modell}. Ein Modell hat die Grundmenge $\N$ und interpretiert $p$ als Nachfolgerrelation, d.h. als $\{(n, n+1) \mid n \in \N \}$. \\
\f{Genauer:} In Logik ohne Gleichheit hat jedes Modell $\INT$ der vermöge Folien 10-16 von \vl{TIL 15} transformierten Formel $(F_{\infty})_{eq} \land G_{eq}$ unendlich viele (nicht-leere und disjunkte) Äquivalenzklassen bzgl. $eq^{\INT}$, und somit auch eine unendliche Grundmenge. Auch hier existiert ein Modell $(\N, \cdot^N)$ mit $p^N := \{(n,n+1) \mid n \in \N\}$ und $eq^N := =$.
\item Jede erfüllbare Formel der Prädikatenlogik erster Stufe hat ein abzählbares Modell. \\
\LOES \textcolor{green}{Die Aussage ist \f{wahr}} nach dem \f{Satz von Löwenheim-Skolem.}
\item Jede Skolemformel hat höchtens eine Herbrand-Interpretation. \\
\LOES \textcolor{orange}{Die Aussage ist \f{falsch.}} \\
Die Skolemformel gibt nur die Grundmenge und die Interpretation der Funktionssymbole vor, \f{die Interpretation der Prädikatensymbole kann jedoch frei gewählt werden} - und dafür gibt es auch mindestens die Möglichkeiten $\emptyset$ und $\{(\delta,\dots,\delta)\}$ für ein $\delta \in \Delta^{\INT}$.
\item Jede Skolemformel hat mindestens ein Herbrand-Modell. \\
\LOES \textcolor{orange}{Die Aussage ist \f{falsch.}} \\
Die unerfüllbare Skolemformel $p() \land \lnot p()$ \f{hat kein Modell}, und somit auch kein Herbrand-Modell.
\end{enumerate}

\subsubsection*{Aufgabe VII}
Zeigen Sie, dass man das Resolutionsverfahren der Prädikatenlogik erster Stufe auch zum Nachweis von semantischen Konsequenzen nutzen kann, indem Sie die Äquivalenz der folgenden Aussagen nachweisen:
\begin{enumerate}
\item $\Gamma \models F$.
\item $\Gamma \cup \{\lnot F\}$ ist unerfüllbar.
\item $\bigwedge \Gamma \to F$ ist allgemeingültig.
\item $\bigwedge \Gamma \land \lnot F$ ist unerfüllbar.
\end{enumerate}
Hierbei sei $\bigwedge \Gamma = \gamma_1 \land \dots \land \gamma_n$ für $\Gamma = \{\gamma_1, \dots, \gamma_n\}$. \\\\
\LOES Die Äquivalenz der vier Aussagen folgt durch einfache oder bereits bekannte Umformungen:
\begin{itemize}[leftmargin=2cm]
\item[$(1)\Leftrightarrow(3)$] gilt nach Aufgabe II (b).
\item[$(3)\Leftrightarrow(4)$] $\bigwedge \Gamma \to F$ ist allgemeingültig gdw. $\lnot (\bigwedge \Gamma \to F) = \bigwedge \Gamma \land \lnot F$ unerfüllbar ist.
\item[$(2)\Leftrightarrow(4)$] Es gibt genau dann kein Modell von $\Gamma \cup \{\lnot F\}$, wenn die Konjunktion $\bigwedge \Gamma \land \lnot F$ kein Modell besitzt.
\end{itemize}
Um also nachzuweisen, dass die Formel $F$ eine semantische Konsequenz der Formelmenge $\Gamma$ ist, zeigt man mittels prädikatenlogischer Resolution, dass die Konjunktion $\bigwedge \Gamma \land \lnot F$ unerfüllbar ist, d.h. dass sich aus der entsprechenden Klauselmenge die leere Klausel $[\,]$ herleiten lässt.

    \lstset{
    frame=single,
    basicstyle=\footnotesize,
    %~ backgroundcolor=\color{light-gray},
    rulecolor=\color{gray},
    %~ linewidth=220pt,
    xleftmargin=0.3cm,
    xrightmargin=3cm,
}

\newcommand{\copaq}{\overline{\mbox{\prob{P}\strut}}_{\text{äquiv}}}
\newcommand{\M}{\mathcal{M}}
\newcommand{\POINT}{{\textcolor{red}{* }}}
\subsection*{Musterklausur SS17}
Die Teilantworten, auf welche man einen Punkt erhält, werden mit \POINT gekennzeichnet.
\subsubsection*{Aufgabe 1 (6 Punkte)}
Sei $f: \N \times \N \setminus \{0\} \to \N$ mit $f(x,y)=(x\,mod\,y)$. Geben Sie ein LOOP-Programm an, welches $f$ berechnet. Dabei dürfen Sie die Abkürzungen aus der Vorlesung benutzen. Erläutern Sie Ihr Programm. \\\\
\LOES \\
\begin{tabular}{p{0.5\textwidth} p{0.5\textwidth}}
\begin{lstlisting}
x0 := x1;
LOOP x1 DO  
  IF x0 >= x2 THEN
    x0 := x0 - x2
  END
END
\end{lstlisting}
\begin{lstlisting}
x0 := x1;
LOOP x1 DO  
  x3 := x0 + 1;
  x3 := x3 - x2;
  IF x_3 != 0 THEN
    x_0 := x_0 - x_2
  END
END
\end{lstlisting}
& 
\vspace{1.5cm}
Ein naives Programm implementiert $mod$ durch sukzessives Abziehen des zweiten Arguments.\POINT \newline

Der Vergleich $x_0 \geq x_2$ kann leicht ersetzt werden.\POINT \newline

Korrekte LOOP-Syntax\POINT

\end{tabular}

\subsubsection*{Aufgabe 2 (9 Punkte)}
\label{MUSTER-2}
Siehe auch Übung 4 Aufgabe 4 [\ref{U4-4}] \\\\
\LOES Wir zeigen zuerst $\phalt \leq_m \paq$ und $\phalt \leq_m \copaq$. Für die erste Reduktion sei $\M$ eine Turing-Maschine und $w$ eine Eingabe für $\M$.\POINT Betrachte die Turing-Maschinen $\M_1,\M_2$ mit \\

\begin{tabular}{p{0.5\textwidth} p{0.5\textwidth}}
\POINT $\M_1 =$ Bei Eingabe $x$:
\begin{itemize}[leftmargin=1.75cm]
\item akzeptiere
\end{itemize} 
&
\POINT $\M_2 = $ Bei Eingabe $x$:
\begin{itemize}[leftmargin=1.75cm]
\item simuliere $\M$ auf $w$
\item akzeptiere
\end{itemize} 
\end{tabular} 
Dann gilt
\begin{equation*}
\M \text{ hält auf } w \Leftrightarrow \LANG(\M_2) = \SIGS \Leftrightarrow \LANG(M_2) = \LANG(M_1).\text{\POINT}
\end{equation*}
Da außerdem die Abbildung $f$ mit
\begin{equation*}
f(enc(\M)\#\#enc(w)) = enc(\M_1)\#\#enc(M_2)
\end{equation*}
berechenbar ist, ist $f$ eine Reduktion von $\phalt$ auf $\paq$.\POINT \\\\
Definiert man nun $\M_1'$ als eine Maschine, die jede Eingabe verwirft, dann ist analog die berechenbare Abbildung
\begin{equation*}
f(enc(\M)\#\#enc(w)) := enc(\M_1')\#\#enc(\M_2)
\end{equation*}
Eine Reduktion von $\phalt$ auf $\copaq$.\POINT \\
Angenommen, $\paq$ wäre semi-entscheidbar. Wegen $\phalt \leq_m \copaq$ ist dann auch $\phalt$ co-semi-entscheidbar und damit entscheidbar, Widerspruch!\POINT Analog folgt aus $\phalt \leq_m \paq$ und der Annahme, dass $\paq$ co-semi-entscheidbar ist, dass $\phalt$ auch co-semi-entscheidbar ist, Widerspruch.\POINT \\
Also ist $\paq$ weder semi-entscheidbar, nocht co-semi-entscheidbar. \\\\
Form\POINT
\subsubsection*{Aufgabe 3 (8 Punkte)}
Welche der folgenden Probleme sind unentscheidbar? Begründen Sie Ihre Antwort.
\begin{enumerate}
\item Gegeben eine Turing-Maschine $\M$ über dem Eingabealphabet $\{0,1,\dots,9\}$ und eine Zahl $n$. Hält $\M$ nach höchstens $n$ Schritten bei Eingabe $42$? \\
\LOES \textcolor{green}{Das Problem ist entscheidbar}\POINT. Simuliere dazu $\M$ für $n$ Schritte mit Eingabe $42$ und akzeptiere, falls diese Simulation hält.\POINT Dieses Verfahren hält stets, da die Simulation von $\M$ spätestens nach $n$ Schritten hält.\POINT 
\item Gegeben eine Turing-Maschine $\M$, ist $\LANG(\M)$ unendlich? \\
\LOES \textcolor{orange}{Das Problem ist nicht entscheidbar}\POINT nach dem Satz von Rice \POINT, da die Eigenschaft \glqq $\LANG(\M)$ ist unendlich\grqq eine nicht-triviale Eigenschaft unendlicher Sprachen ist: $\emptyset$ erfüllt sie nicht, aber $\SIGS$ schon.\POINT
\item Gegeben eine Turing-Maschine $\M$ über einem einelementigen Eingabealphabet, erkennt $\M$ nur Palindrome? \\
\LOES \textcolor{green}{Das Problem ist entscheidbar}\POINT, da jede Maschine über einem einelementigen Eingabealphabet nur Palindrome akzeptiert. Ein Entscheidungsverfahren prüft also nur, ob die Eingabe eine gültige Kodierung einer Turing-Maschine über einem einelementigen Alphabet ist und akzeptiert dann.\POINT
\end{enumerate}

\subsubsection*{Aufgabe 4 (6 Punkte)}
Zeigen Sie: ist $P = NP$, dann gibt es einen Algorithmus, der in polynomieller Zeit für jede erfüllbare aussagenlogische Formel eine erfüllende Belegung findet. \\\\
\LOES Sei $\phi$ eine erfüllbare Formel mit Variablen $x_1,\dots,x_n$.\POINT Betrachte die Formeln $\phi[x_1/\top]$ und $\phi[x_1/\bot]$ und wähle diejenige aus, die erfüllbar ist.\POINT Verfahre analog mit $x_2,\dots,x_n$, bis alle Variablen entweder mit $\top$ oder $\bot$ ersetzt worden sind.\POINT Die entsprechende Wertzuweisung ist dann eine erfüllende Belegung für $\phi$.\POINT \\
Wegen der Annahme $P = NP$ ist der Test auf Erfüllbarkeit in deterministischer polynomieller Zeit realisierbar\POINT, und daher läuft auch dieses Verfahren in polynomieller Zeit ab.\POINT

\subsubsection*{Aufgabe 5 (10 Punkte)}
Wir betrachten folgendes Entscheidungsproblem: Gegeben eine aussagenlogische Formel $F$, gibt es eine erfüllende Belegung von $F$, die nicht alle Variablen wahr macht? Formalisieren Sie dieses Problem als eine Sprache \prob{NAT-SAT} (\textit{not-all-true satisfiability}) und zeigen Sie, dass \prob{NAT-SAT} $NP$-vollständig ist. \\\\
\LOES Wir formalisieren zuerst die Sprache zu 
\begin{align*}
\prob{NAT-SAT} := \{enc(\phi) \mid & \phi \text{ aussagenlogische Formel, die eine erfüllende Belegung hat, } \\
& \text{ in der nicht alle Variablen auf true gesetzt sind. } \}\POINT
\end{align*}
Dann ist $\prob{NAT-SAT} \in NP$\POINT, da eine solche Belegung geraten und in polynomieller Zeit überprüft werden kann.\POINT \\
Um zu zeigen, dass \prob{NAT-SAT} auch NP-vollständig\POINT ist, reduzieren wir in polynomieller Zeit \prob{SAT} auf \prob{NAT-SAT}.\POINT Sei dazu $\phi$ eine aussagenlogische Formel. Definiere
\begin{equation*}
f(enc(\phi)) := enc(\psi)\POINT
\end{equation*}
 mit $\psi := \phi \land \neg x$, wobei $x$ eine neue Variable ist.\POINT Dann ist $\phi$ erfüllbar genau dann, wenn $\psi$ erfüllbar ist mit einer Belegung, in der nicht alle Variablen wahr sind.\POINT Außerdem ist $f$ in polynomieller Zeit berechenbar.\POINT Also ist $\prob{SAT} \leq_p \prob{NAT-SAT}$ und damit \prob{NAT-SAT} auch NP-vollständig.\\\\
Form\POINT 
\subsubsection*{Aufgabe 6 (9 Punkte)}
\begin{enumerate}
\item Bestimmen Sie die Skolemform für folgende Formeln $F$ und $G$.
Geben Sie als Zwischenschritte die bereinigte Form, die Negationsnormalform und die Pränexform an.
\begin{enumerate}[label=\roman*)]
\item $F = \exists x.p(x,y) \to \exists x.q(x,x)$ \\
\LOES 
\begin{align*}
F &= \exists x.p(x,y) \to \exists x.q(x,x) \\
&\equiv\, \neg \exists x.p(x,y) \lor \exists x.q(x,x) \\
&\equiv\, \forall x.\neg p(x,y) \lor \exists x.q(x,x) \qquad &\text{(NNF)\POINT} \\
&\equiv\, \forall x_1.\neg p(x_1,y) \lor \exists x_2.q(x_2,x_2) \qquad &\text{(bereinigt)\POINT} \\
&\equiv\, \forall x_1.\exists x_2.(\neg p(x_y,y) \lor q(x_2,x_2)) \qquad &\text{(Pränexform)\POINT} \\
&\stackrel{Skolem}{\rightarrow}\, \forall x_1.(\neq p(x_y, y) \lor q(f(x_1), f(x_1))) \qquad &\text{(Skolem)\POINT} 
\end{align*}
mit $f$ einem neuen Funktionssymbol. \\
\item $G = \forall x.(\forall y.\exists z.p(x,y,z) \land \exists z.\forall y.\neg p(x,y,z))$ \\
\LOES 
\begin{align*}
G &= \forall x.(\forall y.\exists z.p(x,y,z) \land \exists z.\forall y.\neg p(x,y,z)) \qquad &\text{(bereits in NNF)\POINT} \\
&\equiv\, \forall x.(\forall y.\exists z.p(x,y,z) \land \exists u.\forall v.\neg p(x,v,u)) \qquad &\text{(bereinigt)\POINT} \\
&\equiv\, \forall x.\forall y.\exists z.\exists u.\forall v.(p(x,y,z) \land \neg p(x,v,u)) \qquad &\text{(Pränexform)\POINT} \\
&\stackrel{Skolem}{\rightarrow}\, \forall x.\forall y.\forall v.(p(x,y,f(x,y)) \land \neg p(x,v,g(x,y))) \qquad &\text{(Skolem)\POINT} 
\end{align*}
mit $f,g$ neuen Funktionssymbolen.\\\\
\end{enumerate}
\item Welche der Umformungen sind nicht semantisch äquivalent? Begründen Sie Ihre Antwort. \\
\LOES Bis auf die Skolemisierung sind alle Umformungen semantisch äquivalent.\POINT 
\end{enumerate}

\subsubsection*{Aufgabe 7 (10 Punkte)}
Gegeben sind die prädikatenlogischen Klauseln $K_1$ und $K_2$ mit 
\begin{align*}
K_1 &= \{p(x,f(y)), \neg q(f(x)), \neg q(y)\}, \\
K_2 &= \{\neg p(f(u), f(u)), q(f(v))\}.
\end{align*}
\begin{enumerate}
\item Berechnen Sie \textit{alle} prädikatenlogischen Resolventen von $K_1$ und $K_2$. Erläutern Sie Ihre Vorgehensweise. \\
\LOES Resolventen:
\begin{align*}
&\{\neg q(f(f(u))), \neg q(u), q(f(v)) \}\POINT \qquad &\{x \mapsto f(u), y \mapsto u\}\POINT \\
&\{p(v,f(y)), \neq q(y), \neq p(f(u),f(u)) \}\POINT \qquad &\{x \mapsto v\}\POINT \\
&\{p(x,f(f(v))), \neq q(f(x)), \neq p(f(u),f(u)) \}\POINT \qquad &\{y \mapsto f(v)\}\POINT \\
&\{p(v,f(f(v))), \neq p(f(u),f(u)) \}\POINT \qquad &\{x \mapsto v, y \mapsto f(v)\}\POINT \\
\end{align*}
\item Ist $\{K_1, K_2\}$ erfüllbar? Begründen Sie Ihre Antwort. \\
\LOES Die Klauselmenge $\{K_1, K_2\}$ ist erfüllbar.\POINT Ein Modell ist $\INT = (\Delta^{\INT}, p^{\INT}, q^{\INT}, f^{\INT})$ mit 
\begin{align*}
\Delta^{\INT} &:= \{\delta\}, \\
p^{\INT} &:= \{(\delta,\delta)\}, \\
q^{\INT} &:= \{\delta\},
f^{\INT}(\delta) &:= \delta
\end{align*}
Dann gilt für alle $a,b,c \in \Delta^{\INT}$
\begin{align*}
\INT, \{x \mapsto a, y \mapsto b\} &\models p(x,f(y)),\\
\INT, \{w \mapsto c\} &\models q(f(w)),
\end{align*}
d.h. die beiden Klauseln sind erfüllt und $\INT$ ist in der Tat ein Modell von $\{K_1, K_2\}$.\POINT
\end{enumerate}

\subsubsection*{Aufgabe 8 (16 Punkte)}
Welche der folgenden Aussagen sind jeweils wahr oder falsch? Begründen Sie Ihre Antworten.
\begin{enumerate}
\item Ist $P = NP$, dann ist auch $NP = coNP$. \\
\LOES Ja\POINT, da in diesem Fall $coNP = coP = P = NP$ ist.\POINT
\item Sind $A$ und $B$ Sprachen mit $A \leq_m B$ und $A$ semi-entscheidbar, dann ist auch $B$ semi-entscheidbar. \\
\LOES Nein\POINT, zum Beispiel ist $\prob{SAT} \leq_m \paq$ und $\paq$ ist nicht semi-entscheidbar (siehe Übung 4 [\ref{U4-4}] oder Aufgabe 2 [\ref{MUSTER-2}]).\POINT
\item Die Mengen der Instanzen des Postschen Korrespondenzproblems, welche keine Lösung habhen, ist semi-entscheidbar. \\
\LOES Nein\POINT, da sonst das Postsche Korrespondenzproblem semi-entscheidbar und co-semi-entscheidbar und damit entscheidbar wäre.\POINT
\item Jede kontextfreie Sprache ist auch co-semi-entscheidbar. \\
\LOES Ja\POINT, da alle kontextfreien Sprachen entscheidbar sind.\POINT
\item Es gibt QBF-Formeln, die erfüllbar, aber nicht allgemeingültig sind. \\
\LOES Nein\POINT. Da laut Vorlesung QBF-Formeln niemals freie Variablen haben, fallen Erfüllbarkeit und Allgemeingültigkeit zusammen.\POINT
\item Ist $F$ eine prädikatenlogische Formel mit freien Variablen $x_1,\dots,x_n$, dann ist $F$ genau dann erfüllbar, wenn $\forall x_1,\dots,x_n.F$ erfüllbar ist. \\
\LOES Nein\POINT, zum Beispiel ist $F = \exists x_1, x_2.\neg x_1 \approx x_2 \land y \approx z$ erfüllbar, $\forall y,z.F$ aber nicht.\POINT
\item Ist das Rucksackproblem, bei dem alle Zahlen unär kodiert werden, NP-vollständig,\\
dann ist $P \not= NP$. \\
\LOES Nein\POINT, da das Rucksackproblem mit unär kodierten Zahlen ein Entscheidungsproblem in P ist und dann $P = NP$ folgen würde.\POINT
\item Ist $F$ eine prädikatenlogische Formel ohne Variablen, dann ist $F$ erfüllbar. \\
\LOES Nein\POINT, zum Beispiel ist $F = p(c) \land \neg p(c)$ eine Formel ohne Variablen, die unerfüllbar ist.\POINT
\end{enumerate}

\end{document}
