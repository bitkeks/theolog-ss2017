\subsection*{Übung 8 (Komplexitätstheorie)}
\subsubsection*{Aufgabe 1}
    \begin{enumerate}
        \item Ist Ist $P = \NP$, dann ist $\NP = \MT{coNP}$. \\
            \LOES Da deterministischer Automat für Komplement einfach invertiert werden kann gilt $P = coP$. Demnach $\NP = P = coP = \MT{coNP}$.

        \item Ist $P \neq \NP$, dann gilt $P \neq \MT{coNP}$, $L \neq \NP$ und $P \neq \MT{PSpace}$. \\
            \LOES Für $P = \MT{coNP}$ müsste gelten $P = coP = \MT{cocoNP} = NP$. \\
            Für $L = \NP$: $L \subseteq P \subseteq \NP \Rightarrow P = \NP$ \\
            Für $P = \MT{PSpace}$: $P \subseteq \NP \subseteq \MT{PSpace} \Rightarrow P = \NP$
    \end{enumerate}


\subsubsection*{Aufgabe 2}
    %~ Wir betrachten folgendes Scheduling-Problem: gegeben sind Prüfungen P 1 , ... , P k und
%~ Studierende S 1 , ... , S ` , so dass jede Prüfung von einer bestimmten Menge von Studierenden
%~ abgelegt wird. Die Aufgabe ist, die Prüfungen so in Zeitslots zu legen, dass niemand zwei
%~ Prüfungen im selben Zeitslot ablegen muss. Formalisieren Sie die Frage, ob solch ein
%~ Prüfungsplan mit höchstens h Zeitslots möglich ist, als eine formale Sprache und zeigen Sie,
%~ dass diese NP-vollständig ist. Nutzen Sie dazu die Tatsache, dass Färbbarkeit von Graphen
%~ NP-vollständig ist.

    Scheduling-Problem $SP$, Prüfungen $P_{1}, \dots, P_{k}$, Studierende $S_{1}, \dots, S_{l}$, maximale Anzahl Zeitslots $h$. \\
    Zeigen: Färbbarkeit von Graphen NP-vollständig. \\

    \LOES \\
    Formalisierung des Problems: \hili{$SP = \{\ enc(P, S, f, h)\ |\ \text{es existiert ein Plan für $(P, S, f)$ mit $h$ Zeitslots}\}$}. Hierbei Funktion $f: P \mapsto \POT(S)$, Zuordnung Studierende auf jeweilige Prüfungen.

    $SP \in \NP$. Rate Plan $g: P \mapsto \{\ 1, \dots, h\ \}$. Prüfe, ob Plan gültig.

    SP ist NP-hart. Angewandt auf Färbbarkeit: $\MT{F"arb} = \{\ enc(G, k)\ |\ G \text{ ist $k$-färbbar } \}$

    $\MT{3SAT} \leq_{p} \MT{F"arb} \leq_{p} SP$. Gesucht: $enc(G,k) \to enc(P,S,f,h)$. $G = (V,E) \to (V,E,f,k)$ \\
    $f(v) = \{\ (u_{1}, u_{2}) \in E\ |\ u_{1} = v \text{ oder } u_{2} = v\ \}$.

    $g(w) =
    \begin{cases}
        enc(V,E,f,k) \text{ falls } w = enc(G,k) \\
        \epsilon sonst
    \end{cases}$

    Reduktion: \hili{$G$ ist $k$-färbbar $\Leftrightarrow$ es existiert ein Plan für $(V,E,f)$ mit $k$ Zeitslots.}

    „$\Rightarrow$“ $\exists h$: $V \to \{\ 1, \dots, k\ \}$ gültige Färbung von $G$. $\to h$ ist Plan für $(V,E,f)$, da $P_{1} \neq P_{2}$. $S \in f(P_{1})$ und $S \in f(P_{2})$. $\Rightarrow S = (P_{1}, P_{2})$ oder $S=(P_{2}, P_{1})$. $\Rightarrow h(P_{1}) \neq h(P_{2})$

    „$\Leftarrow$“ $\exists h$: $V \to \{\ 1, \dots, k\ \}$ gültiger Plan für $(V,E,f)$ $\Rightarrow h$ ist gültige Färbung für $G$, da $(u,v) \in E$ (wobei $(u,v) = e$)
        $\Rightarrow e \in f(v)$ und $e \in f(u)$ $\Rightarrow h(v) \neq h(u)$ .

    Hier: Übersetzung eines Färbbarkeitsproblems in ein $SP$. D.h. es existiert nur der Fall, „Student schreibt zwei Prüfungen“ ($(u_{1}, u_{2}) \in E$).
    $SP \leq_{p} \MT{F"arb}$ würde ein allgemeines Problem $SP$ mit beliebigen Prüfungen auf ein $\MT{F"arb}$ reduzieren (komplexer, nicht immer möglich).


\subsubsection*{Aufgabe 3}
    Zeigen Sie, dass folgendes Problem unentscheidbar ist: gegeben eine Turing-Maschine $M$ und eine Zahl $k \in \N$, ist $M$ eine $O(n^{k})$-zeitbeschränkte Turing-Maschine? \\

    \LOES \\
    $\prspec{poly} = \{\ enc(M) \#\# enc(k)\ |\ M \text{ läuft in } O(n^{k})\ \}$. Wobei $n$ Länge der Eingabe.\\
    Wir suchen nun die Reduktion $\phalt \leq_{m} \prspec{poly}$ um \hili{$M$ hält auf $w \Leftrightarrow M'$ läuft in $O(n^{2})$} zu konstruieren. \\
    TM: $enc(M) \#\# enc(w) \mapsto enc(M') \#\# enc(2)$ (hier $k = 2 \in \N$ fest gewählt). \\

    $M'(U) = $ simuliere $M$ auf $w$ in $|U|^{2}$ Schritten ($|U| = n$).
    \begin{itemize}
        \item Wenn $M$ hält auf $w$ dann akzeptiere
        \item Sonst loope für $|U|^{3}$ Schritte und akzeptiere
    \end{itemize}
