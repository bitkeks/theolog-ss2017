\subsection*{Übung 7 (Komplexitätstheorie)}
\subsubsection*{Aufgabe 1}
    Zeigen Sie, dass PSpace unter Komplement, Durchschnitt, Vereinigung, Konkatenation und Kleene-Stern abgeschlossen ist. \\

    \LOES \\
    \hili{PSpace = es gibt eine DTM, die in polynomiellem Platz ein Problem entscheidet.} (kann auch Loop beinhalten)
    \begin{enumerate}
        \item Durchschnitt: sind $L_{1}, L_{2} \in PSpace$ dann auch $L_{1} \cap L_{2} \in PSpace$. Seien dazu $M_{1}$ und $M_{2}$ zwei platzbeschränkte DTM, die $L_{1}$ bzw. $L_{2}$ entscheiden.
            Definiere M = bei Eingabe $w$
            \begin{itemize}
                \item simuliere $M_{1}$ auf $w$; verwerfe falls Simulation verwirft
                \item simuliere $M_{2}$ auf $w$; verwerfe falls Simulation verwirft
                \item akzeptiere
            \end{itemize}
            Dann ist $M$ polynomiell-platzbeschränkt, entscheidet $L_{1} \cap L_{2}$ und ist deterministisch.

        \item Vereinigung: wie Durchschnitt, nur „verwirft“ und „akzeptiere“ vertauscht.

        \item Komplement: $L \in PSpace$, $L = L(M)$, $M$ determinischer polynomiell-platzbeschränkter Entscheider.
            Konstruiere neue Maschine $M'$ durch Vertauschen der Final- und Nichtfinalzustände. Dann ist $M'$ ein deterministisch polynomiell-platzbeschränkter Entscheider für $\SIGS \setminus L$ also $\SIGS \setminus L \in PSpace$.

        \item Konkatenation: $L_{1}, L_{2} \in PSpace \Rightarrow L_{1} \circ L_{2} \in PSpace$ \\
            $L_{1} = L(M_{1})$, $L_{2} = L(M_{2})$

            $M =$ bei Eingabe $w$
            \begin{itemize}
                \item Für alle Belegungen $w = w_{1} w_{2}$ (Raten einer Zerlegung ist NPSpace, laut Savitch gleich PSpace)
                \item Simuliere $M_{1}$ auf $w_{1}$
                \item Simuliere $M_{2}$ auf $w_{2}$
                \item Akzeptiere, falls beide akzeptieren
                \item Verwerfe, sonst
            \end{itemize}

        \item Kleene-Stern: wie Konkatenation, nur mit $w=w_{1},\dots,w_{n}$ (wobei $n$ beliebig, maximal $|w|$), d.h. alle Teilstrings durchprobieren mit maximal $n$ DTM.
    \end{enumerate}


\subsubsection*{Aufgabe 2}
    Gomoku ist auch bekannt als „Fünf gewinnt“.

    %~ Wir betrachten das japanische Spiel Gomoku, welches von zwei Spielern X und O auf einem
    %~ 19×19-Brett gespielt wird. Die Spieler setzen abwechselnd ihre Steine auf das Brett, und
    %~ derjenige Spieler, der zuerst fünf Steine in einer Reihe (horizontal, vertikal, oder diagonal)
    %~ gelegt hat, gewinnt. Spieler X beginnt.
    %~ Verallgemeinertes Gomoku wird statt auf einem Brett fester Größe auf einem beliebigen
    %~ n×n-Brett gespielt. Eine Position in diesem Spiel ist eine Belegung der Felder des Spielbretts
    %~ 1mit Steinen der Spieler X und O, wie sie in einem wirklichen Spiel auftreten könnte. Sei

    \boxed{GM := \{\ enc(B)\ |\ B \text{ ist eine Position im verallgemeinerten Gomoku, in der X eine Gewinnstrategie hat } \}}, wobei $enc(B)$ die zeilenweise Kodierung der Position $B$ über einem festen Alphabet ist. Zeigen Sie $GM \in PSpace$. \\

    \LOES Wir verwenden hierzu einen Baum, der vom Knotenpunkt B abgehend die möglichen Züge von X darstellt. Diese erste Ebene von Positionen, welche nach dem entsprechenden X-Zug entstehen, ist mit dem Existenzquantor markiert. Abgehend von einer Position gibt es nun weitere Baumblätter für die Züge von O, markiert mit dem Allquantor. Danach wieder X/$\exists$ und so weiter. All diese Positionen bedürfen maximal $\leq n^{2}$ Platz.
    \begin{itemize}
        \item Rekursive Definition finden für „gewinnbare Position“
            \begin{itemize}
                \item ist die Position für X bereits gewonnen, dann ist sie gewinnbar
                \item wenn X am Zug, dann ist Position gewinnbar, falls es einen Zug für X gibt, der in einer gewinnbaren Position mündet
                \item wenn O am Zug, dann ist Position gewinnbar, falls alle Züge zu einer gewinnbaren Position führen
            \end{itemize}

        \item Rekursive Auswertung, ob die Wurzel B gewinnbar ist mittels "Tiefensuche". Dies benötigt höchstens so viel Platz, wie der längste Ast im Spielbaum von B lang ist und das ist höchstens $O(n^{2} \cdot log\ n)$.

    \end{itemize}


\subsubsection*{Aufgabe 3}
    Welche der folgenden QBF-Formeln (quantified boolean formula) sind erfüllbar? Begründen Sie Ihre Antwort.
    \begin{enumerate}
        \item $W(\exists p_{1}.p_{1}) = W(p_{1}[p_{1}/\top] \lor W(p_{1}[p_{1}/\bot]))${}
        \item $W(\forall p_{1}.p_{1}) = \dots \land \dots = 1 \land 0 = 0$
        \item $W(\exists p_{1}.\bot) = 0$
        \item $\forall p_{1}.\exists p_{2}. p_{2} \to p_{1}$: Baum zeichnen, ist erfüllbar
        \item $\forall p_{1}. \exists p_{2}. \forall p_{3}. (p_{1} \lor p_{2}) \land p_{3}$: Baum zeichnen
        \item Trivial, nur $\lor \neg p_{3}$ betrachten.
    \end{enumerate}


\subsubsection*{Aufgabe 4}
    \textit{Diese Aufgabe war optional und wurde in der Übung nicht behandelt. Es gibt eine Musterlösung.}


%~ a) ∃p 1 .p 1
%~ b) ∀p 1 .p 1
%~ c) ∃p 1 .⊥
%~ d) ∀p 1 .∃p 2 .p 2 → p 1
%~ e) ∀p 1 .∃p 2 .∀p 3 .(p 1 ∨ p 2 ) ∧ p 3
%~ f) ∀p 1 .∀p 2 .∃p 3 .∀p 4 .(p 1 ∧ p 2 → p 4 ) ∨ ¬p 3
%~ Aufgabe 4
%~ Ein linear bounded automaton (LBA) ist eine deterministische Turing Maschine M , die bei
%~ jeder Berechnung niemals mehr Platz benutzt als bereits durch die Eingabe belegt ist. Dies
%~ erreicht M dadurch, dass sie niemals ein überschreibt und den Lesekopf nach links bewegt,
%~ sobald sie ein liest.
%~ Zeigen Sie, dass das Wortproblem für LBA
%~ P LBA := { enc(M)## enc(w ) | M ein LBA, der w akzeptiert }.
%~ ein PSpace-vollständiges Problem ist.
