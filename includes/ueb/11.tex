\documentclass[a4paper,10pt]{article}

\usepackage[utf8]{inputenc}
\usepackage[ngerman]{babel}
\usepackage[top=2.5cm,bottom=2.5cm,left=2.5cm,right=2.5cm]{geometry}
\usepackage[T1]{fontenc}
\usepackage{graphicx}
\usepackage[table]{xcolor}
\usepackage{fancyhdr}
\usepackage{tgpagella}
\usepackage{marginnote}
\usepackage{nameref}
\usepackage{amsmath,amssymb,amsfonts}
\usepackage{enumitem}
\usepackage{sectsty}
\usepackage{wrapfig}
\usepackage{listings}

\usepackage[
    pdfauthor={Dominik Pataky},
    pdftitle={Theoretische Informatik und Logik},
    pdfsubject={Notizen und Begriffsklärungen aus dem Sommersemester 2017, TU Dresden},
    colorlinks=true,linkcolor=black,urlcolor=link]{hyperref}

\definecolor{light-gray}{gray}{0.9}
\definecolor{light-red}{RGB}{255,105,105}
\definecolor{vl}{RGB}{84,200,70}
\definecolor{link}{RGB}{84,100,220}
\definecolor{sectionblue}{RGB}{0,64,114}
\definecolor{subsectionblue}{RGB}{0,94,167}
\definecolor{subsubsectionblue}{RGB}{0,110,195}

% set enumeration style
\setlist[enumerate, 1]{label=\textbf{\alph*)}, leftmargin=2em}
\setlist{itemsep=0em}

% Small line under top header
\renewcommand{\headrulewidth}{0.1pt}

% Formatting macros
\newcommand{\vl}[1]{\colorbox{vl}{\textcolor{white}{\small\textbf{#1}}}}
\newcommand{\f}[1]{\textbf{#1}}
\newcommand{\blank}{\text{\textvisiblespace}}
\newcommand{\hili}[1]{\colorbox{light-gray}{\textcolor{black}{#1}}}
\newcommand{\verweis}[1]{\textit{\ref{#1} \nameref{#1}}}
\newcommand{\LOES}{\f{Lösung:~}}
% Problems
\newcommand{\prob}[1]{\textbf{#1}}
\newcommand{\prspec}[1]{\prob{P}_{\text{#1}}}
\newcommand{\phalt}{\prspec{halt}}
\newcommand{\paq}{\prspec{äquiv}}
% Math sets
\newcommand{\N}{\mathbb{N}}
\newcommand{\Q}{\mathbb{Q}}
\newcommand{\R}{\mathbb{R}}
\newcommand{\POT}{\mathcal{P}}
% Shortcuts
\newcommand{\TMM}[1]{\mathcal{M}_{#1}}
\newcommand{\LANG}{\mathcal{L}}
\newcommand{\SIGS}{\Sigma^{*}}
% Math texts
\newcommand{\MT}[1]{\mathit{#1}}
\newcommand{\NP}{\MT{NP}}


\begin{document}

\subsection*{Übung 10 (Allgemeinster Unifikator, Resolution)}
\subsubsection*{Aufgabe 1}
Bestimmen Sie jeweils einen allgemeinsten Unifikator der folgenden Gleichungsmengen, oder begründen Sie, warum kein allgemeinster Unifikator existiert. Verwenden Sie hierfür den Algorithmus aus der Vorlesung. Dabei sind $x,y$ Variablen und $a,b$ Konstanten.
\begin{enumerate}
\item $\{ f(x) \dot{=} g(x,y), y \dot{=} f(a) \}$ \\
\LOES Der Algorithmus besteht aus 4 Regeln: 
\begin{itemize}
\item Löschen: $t=t$
\item Orientieren: $t=x \mapsto x=t$
\item Zerlegen: $f(t_1,\dots,t_n) = f(s_1,\dots,s_n) \mapsto t_1=s_1,\dots, t_n=s_n$
\item Einsetzen (Eliminieren): $x=t$ in anderer Gleichung einsetzen, falls $x$ nicht vorkommt.
\end{itemize}
\begin{equation*}
\{ f(x) = g(x,y), y = f(a) \} \rightarrow_{einsetzen} \{ f(x) = g(x,f(a)), y = f(a) \}
\end{equation*}
Keine weitere Regel anwendbar und Menge nicht in gelöster Form. \\
Also gibt es keinen (allgemeinsten) Unifikator.
\item $\{ f(g(x,y)) \dot{=} f(g(a,h(b))) \}$ \\
\LOES 
\begin{equation*}
\{ f'(g(x,y)) = f'(g(a,h(b))) \} \rightarrow_{zerlegen} \{g(x,y) = g(a,h(b)) \rightarrow_{zerlegen} \{x=a, y=h(b)\}
\end{equation*}
Keine weitere Regel anwendbar und Menge in gelöster Form. \\
Ein allgemeinster Unifikator ist $\{x \mapsto a, y \mapsto h(b) \}$
\item $\{ f(x,y) \dot{=} x, y \dot{=} g(x) \}$ \\
\LOES
\begin{equation*}
\{ f(x,y) = x, y = g(x) \} \rightarrow_{orientieren} \{x=f(x,y), y=g(x)\} \rightarrow_{einsetzen} \{x = f(x,g(x)), y= g(x) \}
\end{equation*}
Keine weitere Regel anwendbar. \\
Menge nicht in gelöster Form.
\item $\{ f(g(x),y) \dot{=} f(g(x),a), g(x) \dot{=} g(h(a)) \}$ \\
\LOES
\begin{equation*}
\{ f(g(x),y) = f(g(x),a), g(x) = g(h(a)) \} \rightarrow_{zerlegen} \{ g(x) = g(x), y=a, g(x) = g(h(a)) \} \rightarrow_{zerlegen} \{ g(x) = g(x), y=a, x=h(a) \} \rightarrow_{löschen} \{y=a, x=h(a) \}
\end{equation*}
Fertig, in gelöster Form, ein allg. Unifikator ist $\{y \mapsto a, x \mapsto h(a) \}$.
\item Zusatz. $\{x\dot{=}a, x\dot{=}h(a)\}$  \\
\LOES
\begin{equation*}
\{x=a, x=h(a)\} \rightarrow_{einsetzen} \{h(a) = a, x=h(a) \}
\end{equation*}
Nicht in gelöster Form.
\item Zusatz. $\{x\dot{=}z, y\dot{=}h(z)\}$  \\
\LOES
In gelöster Form. Unifikator $\{ x \mapsto z, y \mapsto h(z) \}$
\end{enumerate}
\subsubsection*{Aufgabe 2}
Zeigen Sie mittels prädikatenlogischer Resolution folgende Aussagen: 
\begin{enumerate}
\item Die Aussage \glqq Der Professor ist glücklich, wenn alle seine Studenten Logik mögen\grqq \\
hat als Folgerung \glqq Der Professor ist glücklich, wenn er keine Studenten hat\grqq. \\
\LOES Wir verwenden folgendes Vokabular: $\{\text{glücklich}/1, \text{magLogik}/1, \text{student}/1, \text{prof}(Konstante)\}$ 
\begin{align*}
F_1 &= \forall x.(\text{student}(x) \to \text{magLogik}(x)) \to \text{glücklich}(prof) \\
F_2 &= \neg \exists x.\text{student}(x) \to \text{glücklich}(prof)
\end{align*} 
\underline{Ziel}: Zeige $F_1 \models F_2$. Zeige dazu, $\{F_1, \neg F_2\}$ ist unerfüllbar. \\
\underline{Normalformen}: 
\begin{align*}
F_1 &= \forall x.(\neg\text{student}(x) \lor \text{magLogik}(x) \to \text{glücklich}(prof) \\
&\equiv \neg \forall x.(\neg \text{student}(x) \lor \text{magLogik}(x) \lor \text{glücklich}(prof) \\
&\equiv \exists x.((\text{student}(x) \land \neg \text{magLogik}(x)) \lor \text{glücklich}(prof) \qquad \text{Pränexform} \\
&=_{Skolem} (\text{student}(c) \land \neg \text{magLogik}(c)) \lor \text{glücklich(prof)} \\ 
&\equiv (\text{student}(c) \lor \text{glücklich}(prof)) \land (\neg \text{magLogik}(c) \lor \text{glücklich}(prof)) \\
\neg F_2 &= \neg (\neg \exists x.\text{student}(x) \to \text{glücklich}(prof)) \\
&\equiv \neg (\exists x.\text{student}(x) \lor \text{glücklich}(prof)) \\
&\equiv \forall x.(\neg \text{student}(x) \land \neg \text{glücklich}(prof))
\end{align*}
Klauselmenge: 
\begin{align*}
\{&\{\text{student}(c), \text{glücklich}(prof)\}^{(1)}, \{\neg\text{magLogik}(c),\text{glücklich}(prof)\}^{(2)}, \\ 
&\{\neg\text{student}(x)\}^{(3)}, \{\neg\text{glücklich}(prof)\}^{(4)}\}
\end{align*}
Resolution:
\begin{align*}
(5) &= (1) + (4) \text{ ergibt } \{\text{student}(c)\} \\
(6) &= (3) + (5) \text{ mit Unifikator } \{x \mapsto c\} \text{ ergibt } \bot
\end{align*}
Also ist $\{F_1, \neg F_2\}$ unerfüllbar und es gilt $F_1 \models F_2$
\item Die Formulierung des Barbier-Paradoxons aus Aufgabe 4 von Blatt 9 ist unerfüllbar. \\
\LOES 
\begin{align*}
F &= \forall x.(rasiert(barbier, x) \leftrightarrow \neg rasiert(x,x) \\
&= \forall x.((rasiert(barbier,x) \lor rasiert(x,x)) \land (\neg rasiert(barbier,x) \lor \neg rasiert(x,x)))
\end{align*}
Klauseln:
\begin{equation*}
\{\{rasiert(barbier,x),rasiert(x,x)\}^{(1)}, \{\neg rasiert(barbier, x), \neg rasiert(x,x)\}^{(2)}\}
\end{equation*}
Resolution: $(1) + (2)$ mit Unifikator $\{x \mapsto barbier\}$ ergibt $\bot$
\item In Aufgabe V folgt die letzte Aussage aus den ersten drei. (Zur Vereinfachung darf hier angenommen werden, dass alle Individuen Drachen sind.) \\
\LOES
\begin{align*}
F_1 &:= \forall x.(\forall y.(\text{kind}(x,y) \to \text{fliegen}(y)) \to \text{glücklich}(x)) \\
F_2 &:= \forall x.(\text{grün}(x) \to \text{fliegen}(x)) \\
F_3 &:= \forall x.(\exists y(\text{kind}(y, x) \land \text{grün}(y)) \to \text{grün}(x)) \\
F_4 &:= \forall x.(\text{grün}(x) \to \text{glücklich}(x))
\end{align*}
Klauselmenge (nach umformulieren):
\begin{align*}
\{&\{\text{kind}(x,f(x)), \text{glücklich}(x)\}^{(1)},\{\neg\text{fliegen}(f(x)), \text{glücklich}(x)\}^{(2)}, \\
&\{\neg\text{grün}(x), \text{fliegen}(x)\}^{(3)}, \{\neg\text{kind}(y,x), \neg\text{grün}(y), \text{grün}(x) \}^{(4)}, \\
&\{\text{grün}(c)\}^{(5)}, \{\neg\text{glücklich}(c)\}^{(6)}\}
\end{align*}
Resolution: 
\begin{align*}
(7) = (1)+(4) & \text{ Variante von } (4): \{\neg \text{kind}(z,w), \neg\text{grün}(z),\text{grün}(w)\} \\
& \text{ Unifikator } \{z \mapsto x, w \mapsto f(x)\} \text{ ergibt Resolvente } \{\text{glücklich}(x),\neg\text{grün}(x),\text{grün}(f(x))\} \\
(8) = (5)+(7) &,\{x \mapsto c\} \text{ ergibt } \{\text{glücklich}(c),\text{grün}(f(c))\} \\
(9) = (8)+(6) & \text{ ergibt } \{\text{grün}(f(c))\} \\
(10) = (3)+(9) & \text{ mit } \{x \mapsto f(c)\} \text{ ergibt } \{\text{fliegen}(f(c))\} \\
(11) = (10)+(2) & \text{ mit } \{x \mapsto c\} \text{ ergibt } \{\text{glücklich}(c)\} \\
(12) = (11)+(6) & \text{ ergibt } \bot
\end{align*}
Also gilt $\{F_1, F_2, F_3\} \models F_4$
\end{enumerate}
\end{document}