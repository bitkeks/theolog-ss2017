\documentclass[a4paper,10pt]{article}

\usepackage[utf8x]{inputenc}
\usepackage[ngerman]{babel}
\usepackage[top=2.5cm,bottom=2.5cm,left=2.5cm,right=2.5cm]{geometry}
\usepackage[T1]{fontenc}
\usepackage{graphicx}
\usepackage{color}
\usepackage{xcolor}
\usepackage{fancyhdr}
\usepackage{tgpagella}
\usepackage[hidelinks]{hyperref}

\title{Notizen VL Theoretische Informatik und Logik}
%\subtitle{}
\author{Dominik Pataky}
\date{\today}

\definecolor{light-gray}{gray}{0.7}

\newcommand{\vl}[1]{\colorbox{light-gray}{\textcolor{white}{\textbf{VL #1}}}}

\begin{document}

    % set pagestyle to fancy and set header chapter to lowercase
    \pagestyle{fancy}

    % Header and footer styles
    \lhead{Notizen TheoLog SS 2017 - Dominik Pataky}
    \rhead{\slshape\nouppercase{\leftmark}}
    \cfoot{\thepage}

    % begin document content
    \maketitle


    \begin{abstract}
        Notizen zur Vorlesung \url{https://iccl.inf.tu-dresden.de/web/Theoretische_Informatik_und_Logik_(SS2017)}
    \end{abstract}


    \tableofcontents

    \vfill
    \begin{tabular}{p{3cm} p{10cm}}
        Professor & Prof. Krötzsch \\
        Ort & TU Dresden \\
        Semester & Sommer 2017 \\
    \end{tabular}


    \newpage
    \section{Begriffe}

    \begin{description}
        \item[Chruch-Turing-These] Die Church-Turing-These
        \item[Turingmaschine] DTM und NTM, besteht aus Tupel M mit (endlicher Menge von Zuständen, Eingabealphabet, Arbeitsalphabet, Übergangsfunktion, Startzustand und Menge von akzeptierenden Endzuständen). \vl{1}
        \begin{description}
            \item[Konfiguration]
                der „Gesamtzustand“ einer TM, bestehend aus Zustand, Bandinhalt und Position des Lese-/Schreibkopfs;
                geschrieben als Wort (Bandinhalt), in dem der Zustand vor der Position des Kopfes eingefügt ist
            \item[Übergangsrelation]
                Beziehung zwischen zwei Konfigurationen wenn die TM von der ersten in die zweite übergehen kann
                (deterministisch oder nichtdeterministisch)
            \item[Lauf]
                mögliche Abfolge von Konfigurationen einer TM, beginnend mit der Startkonfiguration; kann endlich oder unendlich sein
            \item[Halten]
                Ende der Abarbeitung, wenn die TM in einer Konfiguration keinen Übergang mehr zur Verfügung hat
        \end{description}
    \end{description}

\end{document}
